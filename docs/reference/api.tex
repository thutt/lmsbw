% Copyright (c) 2012 Taylor Hutt, Logic Magicians Software
%
% This program is free software: you can redistribute it and/or
% modify it under the terms of the GNU General Public License as
% published by the Free Software Foundation, either version 3 of the
% License, or (at your option) any later version.
%
% This program is distributed in the hope that it will be useful, but
% WITHOUT ANY WARRANTY; without even the implied warranty of
% MERCHANTABILITY or FITNESS FOR A PARTICULAR PURPOSE.  See the GNU
% General Public License for more details.
%
% You should have received a copy of the GNU General Public License
% along with this program.  If not, see <http://www.gnu.org/licenses/>.
%
\chapter{Public API Reference}\label{chap:api}

This chapter facilitates the modification and extension of \lmsbw.

\section{\texttt{declare\_source\_component}}\label{api:declare-source-component}

This function allows the declaration of \emph{source-based} components
in a wrapped project.

Source-based means that the entire source for the component is
available in a directory tree that is currently accessible on the
machine running the build.  The source tree must have a \makefile at
the top level that will build the source in an \lmsbw-compliant
manner.  See \xref{wrap:component-makefile} for details on \makefile
compliance.

\begin{minipage}{\linewidth}
\begin{verbatim}
declare_source_component
    <component-name>,
    <description>,
    <reason>,
    <configuration-file>,
    <full path to source directory>,
    <optional list of prerequisite components>
\end{verbatim}

\begin{itemize}
\item{component-name}

  This is the name used to reference the component throughout the
  \lmsbw-project.  The name must be a valid \make identifier, and it
  must be globally unique within the project.

  See \xref{api:component}.

\item{description}

  This is a brief description of the purpose of the component.  The
  description will be printed by various verbs of the \lmsbw system.

  Even though this is a required parameter, the description is purely
  informational.

  See \xref{api:description}.

\item{reason}

  This is the \emph{reason} the component is being included in the
  build.

  See \xref{api:reason}.

\item{configuration file}

  This must be the full pathname of the component configuration file
  which declares the source component.

  See \xref{api:configuration-file}.

\item{source directory}

  The full pathname to the source directory for the component.

  See \xref{api:source-directory}.

\item{optional list of prerequisites}

  If specified, this argument must hold a space-separated list of
  component names.

  See \xref{api:prerequisite}.

\end{itemize}
\end{minipage}


\section{\texttt{component\_attribute\_no\_parallel\_build}}\label{api:component-attribute-no-parallel-build}

This function ensures that the associated component's build is never
able to further invoke recursive instances of \make in parallel.

This attribute is normally not necessary, but can be handy for
components which have a poorly-created \makefile; a build system which
will benefit from this option will intermittently fail when being
built with high levels of parallelism because of incorrect
dependencies.

\begin{minipage}{\linewidth}
\begin{verbatim}
component_attribute_no_parallel_build
    <component-name>
\end{verbatim}

\begin{itemize}
\item component-name

  This is the name of the component for which parallel builds should
  not be allowed.

  The component must have been declared before this attribute can be
  set.

  When this attribute is set, the recursive invocation of \gnumake
  used to build the component will have \texttt{-j 1} explicitly set
  on the command line.

\end{itemize}
\end{minipage}

\section{\texttt{component\_attribute\_build\_target}}\label{api:component-attribute-build-target}

This changes the target for building a component with the component's
build process from the \emph{default} \make target to the list of
targets provided.

\begin{minipage}{\linewidth}
\begin{verbatim}
component_attribute_build_target
    <component-name>
    <list of targets>
\end{verbatim}

\begin{itemize}
\item component-name

  This is the name of the component for which the build target value
  should be set.

  The component must have been declared before this attribute can be
  set.

\item list-of-targets

  This argument is a space-separated list of targets in the
  component's \makefile which will be used for the build phase of the
  component.
\end{itemize}
\end{minipage}

\section{\texttt{component\_attribute\_install\_target}}\label{api:component-attribute-install-target}

This changes the target for installing a component with the
component's build process from the \texttt{install} \make target to
the list of targets provided.

\begin{minipage}{\linewidth}
\begin{verbatim}
component_attribute_install_target
    <component-name>
    <list of targets>
\end{verbatim}

\begin{itemize}
\item component-name

  This is the name of the component for which the install target value
  should be set.

  The component must have been declared before this attribute can be
  set.

\item list-of-targets

  This argument is a space-separated list of targets in the
  component's \makefile which will be used for the install phase of
  the component.

\end{itemize}
\end{minipage}
  \todo{Document build \& install phase ideas.}

\section{\texttt{component\_attribute\_api}}\label{api:component-attribute-api}

This allows the specification of \emph{install} directories that
contain an exported \emph{API}.

All components that list this component as a prerequisite will
automatically become dependent upon changes to the directories named;
if anything changes in those directories, the dependent component is
rebuilt.

It is incumbent upon the dependent component to have proper rules in
the component \makefile to detect changes in these API
directories\footnote{For example, the \texttt{gcc} option
  \texttt{-MMD} combined with the \makefile including the generated
  dependencies}.  In other words, if an API header file or static
library changes, it is up to the dependent \makefile to have the
proper rules to detect this change and rebuild or relink as necessary.

\begin{minipage}{\linewidth}
\begin{verbatim}
component_attribute_api
    <component-name>,
    <list of directories>
\end{verbatim}

\begin{itemize}
\item component-name

  This is the name of the component for which a set of \emph{api}
  directories should be maintained.

  The component must have been declared before this attribute can be
  set.

\item list-of-directories

  This is a space-separated list of directories which contain files
  that can be used by other, dependent, components.

  The contents of each directory contained in this list should be
  filled in \destdir by the \texttt{install} target of the build
  process.

  The following rules are recommended when creating API directories:

  \begin{itemize}
    \item Use root-based pathnames

      Because you are going to install everything into \destdir, and
      \lmsbw proper will copy those files into the install directory,
      you should use root-based API directory names.

    \item Use the name of the component in the directory name.

      This will make it easy for dependent component maintainers to
      easily find your API locations.\footnote{It is also easy to
        execute the \emph{report} verb on a component to find out
        the location of API directories.}

    \item Standardize on install locations

      If you have header files, they should be placed in:

      \texttt{/usr/include/\emph{component-name}}.

      If you have libraries, put them in
      \texttt{/usr/lib/\emph{component-name}}.

      If you have some other type of files, be sure to reach consensus
      on standardized locations; it will make maintenance much easier.

      Since component names must be unique, the directory names will
      also be unique.  And, this rule allows dependent components to
      easily include your header files, and link with your libraries
      -- because their location will be known.

      See \xref{wrap:using-prerequisite-api} for details on using an API
      exported by a prerequisite component.

    \item Install to \destdir

      Install all files into the API directory prefixed with \destdir.

      If you don't prefix with \destdir, your build process will
      likely fail with permission errors on the root-based pathnames.

  \end{itemize}
\end{itemize}
\end{minipage}

\section{\texttt{component\_attribute\_cflags}}\label{api:cflags}

This allows the specification of \texttt{CFLAGS} values which are
passed to the component build process.  It is incumbent upon the
component's \makefile to never directly assign to \texttt{CFLAGS};
rather it should append values to the \lmsbw-provided value.

\lmsbw uses the declared value of \texttt{CFLAGS} in the assignment of
the build directory.

Using the \texttt{CFLAGS} value in the assignment of the build
directory results in being able to have multiple build versions --
say, debugging \& optimized -- of your component at one time.
Switching between different \texttt{CFLAGS} values will always result
in the minimal (from the last build) incremental build for the
component.

\begin{minipage}{\linewidth}
\begin{verbatim}
component_attribute_cflags
    <component-name>,
    <cflags-value>
\end{verbatim}

\begin{itemize}
\item component-name

  This is the name of the component for which the \texttt{CFLAGS}
  value should be set.

  The component must have been declared before this attribute can be
  set.

\item cflags-value

  This the value to which \texttt{CFLAGS} will be set when initiating
  the component's build process.

  Be sure to match the values set with the toolchain that is being
  used.

\end{itemize}
\end{minipage}

\section{\texttt{declare\_component\_kind}}\label{api:kind}

This internal function allows the specification of the \texttt{kind}
of the component.

The value is used internally to generate \make rules to build \&
install the component.

See \xref{variables:kind} for allowed values and semantics.

\begin{minipage}{\linewidth}
\begin{verbatim}
declare_component_kind
    <component-name>,
    <kind>
\end{verbatim}

\begin{itemize}
\item component-name

  This is the name of the component for which the \emph{kind} is to be declared.

  The component must have been declared before this attribute can be
  set.

\item kind

  This provides the value to which the component's \texttt{kind} will
  be set.
\end{itemize}
\end{minipage}

\section{\texttt{declare\_component\_description}}\label{api:description}

This function allows the specification of a component's description.

\begin{minipage}{\linewidth}
\begin{verbatim}
declare_component_description
    <component>,
    <description>
\end{verbatim}

\begin{itemize}
\item component-name

  This is the name of the component for which the component's
  \texttt{description} variable should be set.

  The component must have been declared before this attribute can be
  set.

\item description

  This is the value to which the component's \texttt{description}
  variable will be set.

\end{itemize}
\end{minipage}

\section{\texttt{declare\_component\_reason}}\label{api:reason}

This function allows the specification of the \emph{reason} that a
component is included in the build.

See \xref{variables:reason} for allowed values and semantics.

\begin{minipage}{\linewidth}
\begin{verbatim}
declare_component_reason
    <component>,
    <reason>
\end{verbatim}

\begin{itemize}
\item component-name

  This is the name of the component for which the component's
  \texttt{reason} variable should be set.

  The component must have been declared before this attribute can be
  set.

\item reason

  This is the value to which the component's \texttt{reason} variable
  will be set.

\end{itemize}
\end{minipage}

\section{\texttt{declare\_component\_component}}\label{api:component}

This function allows the specification of the name of the component.

\begin{minipage}{\linewidth}
\begin{verbatim}
declare_component_component
    <component-name>,
    <name>
\end{verbatim}

\begin{itemize}
\item component-name

  This is the name of the component for which the component's
  \texttt{component} variable should be set.

  The component must have been declared before this attribute can be
  set.

\item name

  This is the value to which the component's \texttt{component}
  variable will be set.

\end{itemize}
\end{minipage}

\section{\texttt{declare\_component\_prerequisite}}\label{api:prerequisite}

This function allows the specification of a list of prerequisite
components.

\lmsbw will ensure that all the named components are built and
installed prior to building this component.

The prerequisite components do not need to have been declared before
they are referenced.

If a component is named as a prerequisite that does not actually
exist, \make will issue an error when attempting to fulfill the
targets necessary to produce the build.

If a prerequisite component has declared any API directories
(\xref{api:component-attribute-api}), those API directories are
checked for changes as part of determining if this component needs to
be built.  If any prerequisite component's API has changed files, then
this component will be built.

\begin{minipage}{\linewidth}
\begin{verbatim}
declare_component_prerequisite
    <component>,
    <prerequisites>
\end{verbatim}

\begin{itemize}
\item component-name

  This is the name of the component for which the component's
  \texttt{prerequisite} variable should be set.

  The component must have been declared before this attribute can be
  set.

\item prerequisites

  This is a space-separated list of component names which are
  considered to be prerequisites of this component.

  Each prerequisite will be successfully built \& installed before
  this component is built.
\end{itemize}
\end{minipage}

\section{\texttt{declare\_component\_source\_directory}}\label{api:source-directory}

This function allows the specification of the full pathname to the
source code for the associated component.

This must be the full path of the directory containing the source, and
importantly, it must also contain the \makefile which actually builds
the component.

\lmsbw will produce an error if the named directory does not exist.

\lmsbw expects that a \makefile will be at the top level of the named
directory.

\begin{minipage}{\linewidth}
\begin{verbatim}
declare_component_source_directory
    <component>,
    <absolute-directory>
\end{verbatim}

\begin{itemize}
\item component-name

  This is the name of the component that will have its
  \texttt{source-directory} variable set.

  The component must have been declared before this attribute can be
  set.

\item absolute-directory

  This must be an absolute pathname to the source directory for the
  named component.

  If the directory does not exist, \lmsbw will produce an error.
\end{itemize}
\end{minipage}

\section{\texttt{declare\_component\_configuration\_file}}\label{api:configuration-file}

This function allows the specification of the full pathname of the
configuration file associated with the component.

\lmsbw sets up dependencies so that changes to this file will
automatically force the component's build process to be invoked.

\begin{minipage}{\linewidth}
\begin{verbatim}
declare_component_configuration_file
    <component>,
    <configuration-file>
\end{verbatim}

\begin{itemize}
\item component-name

  This is the name of the component that will have its
  \texttt{configuration-file} variable set.

  The component must have been declared before this attribute can be
  set.

\item configuration-file

  This should be the absolute pathname of the component configuration
  file which declared the component.

\end{itemize}
\end{minipage}

\section{\texttt{component\_attribute\_toolchain}}\label{api:toolchain}

\begin{minipage}{\linewidth}
\begin{verbatim}
component_attribute_toolchain
    <component-name>,
    <toolchain-name>
\end{verbatim}

\begin{itemize}
\item component-name

  This is the name of the component for which the toolchain should be
  set.

  The component must have been declared before this attribute can be
  set.

\item toolchain

  This is the directory name of the toolchain which should be
  configured when building and installing this component.

  If the toolchain does not exist, \lmsbw will produce an
  error.
\end{itemize}

See \xref{chap:toolchain-usage} and
\xref{chap:toolchain-configuration} for information on using and
configuring toolchains.
\end{minipage}
