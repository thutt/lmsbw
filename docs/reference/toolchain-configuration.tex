% Copyright (c) 2012 Taylor Hutt, Logic Magicians Software
%
% This program is free software: you can redistribute it and/or
% modify it under the terms of the GNU General Public License as
% published by the Free Software Foundation, either version 3 of the
% License, or (at your option) any later version.
%
% This program is distributed in the hope that it will be useful, but
% WITHOUT ANY WARRANTY; without even the implied warranty of
% MERCHANTABILITY or FITNESS FOR A PARTICULAR PURPOSE.  See the GNU
% General Public License for more details.
%
% You should have received a copy of the GNU General Public License
% along with this program.  If not, see <http://www.gnu.org/licenses/>.
%
\chapter{Toolchain Configuration}\label{chap:toolchain-configuration}

Configuring toolchains used to be a painful by-hand task that no
reasonable developer would willingly undertake, but now with the
availability of \texttt{crosstool-ng} (aka \ctng), it's a simple matter
to crank out a wide variety of toolchains.

\section{Prerequisites}

Before embarking on an odyssey to build toolchains, you must first
make a few configuration changes for uniformity.  These changes are
not absolutely necessary.  You may skip them if you so desire, but you
must be aware that all future documentation about the toolchains
assumes you have performed these steps.

We will put all the toolchains into a single directory tree, and
allow \ctng to place its downloaded tarballs into a single
directory, too.

\begin{verbatim}
sudo mkdir /opt/crosstool-ng
sudo chown --recursive ${USER}:${USER} /opt/crosstool-ng

sudo mkdir /opt/toolchains
sudo chown --recursive ${USER}:${USER} /opt/toolchains

sudo mkdir /opt/toolchain-source-tarballs
sudo chown --recursive ${USER}:${USER} /opt/toolchain-source-tarballs
\end{verbatim}

\section{\texttt{crosstool-ng}}

This utility, which uses a Linux-like configuration system, handles
all the messy details of fetching, configuring, building and
installing Gnu toolchains.

\begin{description}
\item Download \& install \ctng

  You must download and install \ctng without instructions because
  each host computer can run into a unique set of issues such as
  missing prerequisites like \texttt{yacc}.

  Configure \ctng to install into \texttt{/opt/crosstool-ng}

  Once downloaded, configured and installed, you are ready to
  configure and build a toolchain.

  \item Run \ctng

    \ctng will save its configuration and produce a build in the
    current working directory.  It is recommended that you create a
    directory where you want to store the build output and toolchain
    configuration, and change into that directory.

\begin{verbatim}
mkdir ~/myfirsttoolchain
cd ~/myfirsttoolchain
/opt/crosstool-ng/bin/ct-ng menuconfig
\end{verbatim}

    This will start a Linux-like configuration system.  The following
    items will walk you through the items you must set to produce a
    toolchain.

    Leave unmentioned items unchanged.

    \begin{description}

    \item Paths and misc options

      \begin{itemize}
      \item{Prefix Directory}

        Set to: \texttt{/opt/toolchains/\$\{CT\_TARGET\}}

      \item{Render the toolchain read-only}

        If you will need to make changes to the toolchain, or possibly
        delete it, unselect this option.

        Recommendation: Unselect

      \item{Number of parallel jobs}

        This option affects the number of parallel jobs used when
        building the toolchain.  The system will automatically
        determine a reasonable number, but you can change it.


        Recommendation: Do not change.

      \item{Maximum Allowed load}

        This option affects the total load building the toolchain can put on
        your system.  Adjust for your personal taste.

        Recommendation: Do not change.

      \end{itemize}

    \item{Target Options}
      \begin{itemize}
      \item{Target Architecture}

        Select the desired target architecture.

      \item{Endianess}

        If allowed, select the endianess for the selected architecture.

      \item{Bitness}

        If allowed, select the bitness for the selected architecture.

      \item{ABI}

        If allowed, select the desired ABI for the selected architecture.

      \end{itemize}

    \item{Toolchain Options}

      \begin{itemize}
      \item{Tuple's vendor string}

        This is a string which can be used to help uniquely identify
        the toolchain.  A tuple is of the form
        'arch-vendor-kernel-system' and is used to form the name of
        the executables in the toolchain, as well as the toolchain
        directory name.

        If you intend on creating many different toolchains -- perhaps
        for different compiler versions, or different kernels -- it is
        recommended to create a vendor string which mnemonically
        distinguishes toolchains from one another.

        You could use, for example, the gcc version as the vendor
        string.  For a \emph{bare metal} MIPS configuration, you might
        end up with a tuple like this: \texttt{mipsel-gcc\_4.6.3-elf}.

        You may not use dashes in the vendor string.
      \end{itemize}

    \item{Operating System}
      \begin{itemize}
      \item{Target OS}

        Choose the type of OS you wish to target.

        If you choose \texttt{linux}, you will have to make several
        more choices about the Linux kernel.
      \end{itemize}

    \item{Binary Utilities}

      The defaults here are normally sufficient.

    \item{C compiler}

      Choose the compiler version and the desired additional language
      support.

    \item{C-library}

      Choose the desired C-library implementation.

    \item{Debug facilities}

      Choose the desired debugging facilities.

    \item{Companion Libraries}

      Choose the desired companion libraries.
    \end{description}
  \item Build your toolchain

    The next, and final step in the process of building a toolchain to
    to execute the following:

\begin{verbatim}
/opt/crosstool-ng/bin/ct-ng build
\end{verbatim}

   \item Use your toolchain

     Your toolchain was installed in \texttt{/opt/toolchains}.

     You may access it directly, or you can use the \emph{populate}
     script to install the toolchain, the headers and all the
     libraries in a \emph{sysroot} directory that can be used for
     building.

\end{description}
