% Copyright (c) 2012 Taylor Hutt, Logic Magicians Software
%
% This program is free software: you can redistribute it and/or
% modify it under the terms of the GNU General Public License as
% published by the Free Software Foundation, either version 3 of the
% License, or (at your option) any later version.
%
% This program is distributed in the hope that it will be useful, but
% WITHOUT ANY WARRANTY; without even the implied warranty of
% MERCHANTABILITY or FITNESS FOR A PARTICULAR PURPOSE.  See the GNU
% General Public License for more details.
%
% You should have received a copy of the GNU General Public License
% along with this program.  If not, see <http://www.gnu.org/licenses/>.
%
\chapter{Toolchain Configuration}\label{chap:toolchain-configuration}

Configuring toolchains used to be a painful by-hand task that no
reasonable developer would willingly undertake, but now with the
availability of \texttt{crosstool-ng} (aka \ctng), it's a simple matter
to crank out a wide variety of toolchains.

\section{Prerequisites}\label{toolchain-config:prerequisites}

Before embarking on an odyssey to build toolchains, a few
configuration changes must first be made, for uniformity.  These
changes are not absolutely necessary, but all documentation about the
toolchains assumes these steps have been performed.

\subsection{Directories}
All toolchains and tarballs downloaded by \ctng will be put into a
single directory trees.

\begin{footnotesize}
\begin{verbatim}
sudo mkdir /opt/crosstool-ng
sudo chown --recursive ${USER}:${USER} /opt/crosstool-ng

sudo mkdir /opt/toolchains
sudo chown --recursive ${USER}:${USER} /opt/toolchains

sudo mkdir --parents /opt/toolchain-admin/{tarballs,build}
sudo chown --recursive ${USER}:${USER} /opt/toolchain-admin
\end{verbatim}
\end{footnotesize}

\subsection{Tools}

The following tools are known to be required before configuring and
building \ctng.

\begin{itemize}
\item{subversion}

  Some of the tools that are downloaded by \ctng use subversion.

\end{itemize}

\section{\texttt{crosstool-ng}}\label{toolchain-config:install-ctng}

This utility, which uses a Linux-like configuration system, handles
all the messy details of fetching, configuring, building and
installing Gnu toolchains.

Detailed instructions for the following steps are beyond the scope of
this reference manual.

\begin{enumerate}
\item Download \ctng

\item Configure \ctng to install into \texttt{/opt/crosstool-ng}

\item Build \ctng

\item Install \ctng

\end{enumerate}

Once \ctng has been installed, it will be possible to create
toolchains.

\begin{enumerate}
\item Run \ctng

    \ctng will save its configuration and produce a build in the
    current working directory.  It is recommended to create a
    directory to store the build output and toolchain configuration,
    and change into that directory.

\begin{verbatim}
mkdir ~/myfirsttoolchain
cd ~/myfirsttoolchain
/opt/crosstool-ng/bin/ct-ng menuconfig
\end{verbatim}

    This will start a Linux-like configuration system.  The following
    steps provides a walk through of items to change to produce a
    toolchain.

    Leave unmentioned items unchanged.

    \begin{itemize}

    \item Paths and misc options

      \begin{itemize}
      \item{Local tarballs directory}

        Set to: \texttt{/opt/toolchain-admin/tarballs}

        This will allow sharing of common tarballs among various
        toolchain builds.

      \item{Save new tarballs}

        Select this item so that tarballs will be saved in the
        directory entered above.

      \item{Working Directory}

        Set to: \texttt{/opt/toolchain-admin/build/\$\{CT\_TARGET\}}

        This is the directory where the toolchain will be built.

      \item{Prefix Directory}

        Set to: \texttt{/opt/toolchains/\$\{CT\_TARGET\}}

      \item{Render the toolchain read-only}

        Unselecting this item will ensure that the toolchain is not
        marked \emph{read-only}.

        Recommendation: Unselect

      \item{Number of parallel jobs}

        This option affects the number of parallel jobs used when
        building the toolchain.  The system will automatically
        determine a reasonable number.

        Recommendation: Do not change.

      \item{Maximum Allowed load}

        This option affects the total system load building the
        toolchain can put on the host system.

        Recommendation: Do not change.

      \end{itemize}

    \item{Target Options}
      \begin{itemize}
      \item{Target Architecture}

        Select the desired target architecture.

      \item{Endianess}

        If allowed, select the endianess for the selected architecture.

      \item{Bitness}

        If allowed, select the bitness for the selected architecture.

      \item{ABI}

        If allowed, select the desired ABI for the selected architecture.

      \end{itemize}

    \item{Toolchain Options}

      \begin{itemize}
      \item{Tuple's vendor string}

        Set to a value determined by following the suggestions in
        \xref{toolchain-config:vendor-string}.

        \item{Tuple's alias}

          Set to: \texttt{lmsbw}.

          This will provide a simpler interface for hand-use of the
          toolchains; the main programs in the toolchain will have an
          alias with a simple \texttt{lmsbw-} prefix.

      \end{itemize}

    \item{Operating System}
      \begin{itemize}
      \item{Target OS}

        Choose the type of OS to be the target of the toolchain.

        Choosing \texttt{linux}, will induce several more
        kernel-specific options to appear.
      \end{itemize}

    \item{Binary Utilities}

      The defaults here are normally sufficient.

    \item{C compiler}

      Choose the compiler version and the desired additional language
      support.

    \item{C-library}

      Choose the desired C-library implementation.

    \item{Debug facilities}

      Choose the desired debugging facilities.

    \item{Companion Libraries}

      Choose the desired companion libraries.
    \end{itemize}
  \item Build the toolchain

    The next, and final step in the process of building a toolchain to
    to execute the following:

\begin{verbatim}
/opt/crosstool-ng/bin/ct-ng build
\end{verbatim}

   \item Use the toolchain

     The toolchain was installed in \texttt{/opt/toolchains}.

     It can be accessed directly, or the \emph{populate} script to
     install the toolchain, the headers and all the libraries in a
     \emph{sysroot} directory that can be used for building.

\end{enumerate}

\subsection{Determining Vendor Tuple}\label{toolchain-config:vendor-string}

This is a string that helps uniquely identify the toolchain.  A tuple
is of the form 'arch-vendor-kernel-system' and is used to form the
name of the executables in the toolchain, as well as the toolchain
directory name.  This section describes how to concot the
\emph{vendor} portion of this string.

If there is an intention to create many different toolchains --
different compiler versions, different architectures, different
kernels, etc. -- it is recommended to create a vendor string which
mnemonically distinguishes toolchains from one another so that each
toolchain's attributes are clearly discernable when looking at a
directory listing of toolchains.

The following rules were used when creating the vendor string for the
toolchain samples included in \lmsbw; perhaps they will help in
creating a policy for the vendor string of local toolchains:

\begin{itemize}
\item{Dashes may not be used in the vendor string.}
\item{Use '\_' for inter-section spacing.}
\item{Use '.' for intra-section spacing.}
\item{Use \emph{endianess}}
\item{Use \emph{bit size}}
\item{Use compiler info}
\item{Use kernel / bare metal info}
\item{Use C library information}
\end{itemize}

Here are example applications of these rules; these are the exact
directory names created by the installation of the sample toolchains
provided with \lmsbw:

\begin{description}
\item{\texttt{arm-le\_bare.metal\_gcc.4.6.3\_newlib.1.17.0-eabi}}

  Toolchain producing code for a \emph{bare metal} \emph{little
    endian} \emph{arm} system using \emph{gcc 4.6.3} and the
  \emph{newlib 1.17.0} runtime library.

\item{\texttt{arm-le\_linux.3.3.4\_gcc.4.6.3\_eglibc.2.15-linux-gnuabi}}

  Toolchain producing code for a \emph{Linux 3.3.4} \emph{little
    endian} \emph{arm} system using \emph{gcc 4.6.3} and the
  \emph{eglibc 2.15} runtime library.

\item{\texttt{i386-32\_bare.metal\_gcc.4.6.3\_newlib.1.17.0-elf}}

  Toolchain producing code for a \emph{bare metal} \emph{32-bit}
  \emph{x86} system using \emph{gcc 4.6.3} and the \emph{newlib
    1.17.0} runtime library.

\item{\texttt{mips-be\_bare.metal\_gcc.4.6.3\_newlib.1.17.0-elf}}

  Toolchain producing code for a \emph{bare metal} \emph{big endian}
  \emph{MIPS} system using \emph{gcc 4.6.3} and the \emph{newlib
    1.17.0} runtime library.

\item{\texttt{mips-be\_linux.3.3.4\_gcc.4.6.3\_eglibc.2.15-linux-gnu}}

  Toolchain producing code for a \emph{linux 3.3.4} \emph{big
    endian} \emph{MIPS} system using \emph{gcc 4.6.3} and the
  \emph{eglibc 2.15} runtime library.

\item{\texttt{mipsel-le\_bare.metal\_gcc.4.6.3\_newlib.1.17.0-elf}}


  Toolchain producing code for a \emph{bare metal} \emph{little
    endian} \emph{MIPS} system using \emph{gcc 4.6.3} and the
  \emph{newlib 1.17.0} runtime library.

\item{\texttt{mipsel-le\_linux.3.3.4\_gcc.4.6.3\_eglibc.2.15-linux-gnu}}

  Toolchain producing code for a \emph{linux 3.3.4} \emph{little
    endian} \emph{MIPS} system using \emph{gcc 4.6.3} and the
  \emph{eglibc 2.15} runtime library.

\item{\texttt{x86\_64-64\_bare.metal\_gcc.4.6.3\_newlib.1.17.0-elf}}

  Toolchain producing code for a \emph{bare metal} \emph{64-bit}
  \emph{x86} system using \emph{gcc 4.6.3} and the \emph{newlib
    1.17.0} runtime library.


\item{\texttt{x86\_64-64\_linux.3.3.4\_gcc.4.6.3\_eglibc.2.15-linux-gnu}}

  Toolchain producing code for a \emph{linux 3.3.4} \emph{64-bit}
  \emph{x86} system using \emph{gcc 4.6.3} and the \emph{eglibc 2.15}
  runtime library.
\end{description}
