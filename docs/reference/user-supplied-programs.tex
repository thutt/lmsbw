\chapter{User Supplied Programs}

This chapter describes the programs that must be supplied by the build
team to support certain features of \lmsbw.  The \lmsbw features that
require additional scripts require, for lack of a better phrase,
\emph{business logic} that is unique to each build system using
\lmsbw.


\section{Build Output Download Script}\label{api:build-output-download-script}

The \texttt{component\_attribute\_build\_output\_download} API
(\xref{api:build-output-download}) attributed components will have
their build output downloaded using the script specified with with
configuration key described in
\xref{variables:component-build-output-download-script}.  However, the
script that actually performs the download of the build output must be
supplied by the build team because \lmsbw is not written to
incorporate the configuration specifics of various build systems using
it.  This section describes the work necessary to write a download
script that will work seamlessly with \lmsbw.

Builds that do not use the
\texttt{component\_attribute\_build\_output\_download} API do not need
to provide the script.

\subsection{Implementation Reference}

\begin{itemize}
  \item{Master Configuration}

    To enable the functionality in \lmsbw to download build output,
    the master configuration key
    \texttt{component-build-output-download-script} must be set to the
    name of the program that conforms to the semantics described in
    this section.

    See~\xref{variables:component-build-output-download-script}.

  \item{Component Configuration}

    Each component that is to have its build output downloaded must
    be configured.

    See \xref{variables:build-output-download} to understand the
    per-component variable, and see \xref{api:build-output-download}
    to understand how to set the variable using the public API.
\end{itemize}

\subsection{Script Arguments}

The build output download script will always be invoked with the
following non-optional, string type, command line arguments.

\begin{itemize}
\item{\texttt{--component <component name>}}

  This is the name of the component for which the build output should
  be downloaded.

\item{\texttt{--build-root-directory <pathname>}}

  This argument will provide the full pathname to the root of the
  build directory.  See \xref{variables:build-root-directory}.

  The script may place \emph{helper} files into this directory.  It
  should not be used for intermediate files; \texttt{/tmp} must be used
  for intermediate files.

  This directory is guaranteed to exist.

\item{\texttt{--destdir-directory <pathname>}}

  This argument provides the full pathname of the directory to that
  the build output should be written.

  This directory is guaranteed to exist.

\item{\texttt{--source-directory <pathname>}}

  This argument provides the full pathname of the component's source
  directory.

  This directory is guaranteed to exist.

\item{\texttt{--help}}

  This option must provide information on the allowed command line
  arguments, and then it must exit the program with a return code of
  zero (0).

\item{\texttt{--verbose}}

  When \texttt{--verbose} is provided to \lmsbwcmd, it will be
  provided to build output download script.  When this option is
  provided, the script can provide more verbose output to
  \texttt{stdout}.

  The script is not required to have any additional output simply
  because \texttt{--verbose} is provided on the command line.
\end{itemize}

\subsection{Script Semantics}

\begin{description}
\item[Invocation] \mbox{} \\

  This program is invoked under the conditions shown in
  \xref{intro:build-component-criteria}.

\item[Detect Source Changes] \mbox{} \\

  If the local sources have been changed relative to the source
  actually committed to the revision control system, an error must be
  returned.

\item[Missing Build] \mbox{} \\

  If a build that corresponds to the unmodified sources cannot be
  found, an error must be returned.

\item[Successful Download] \mbox{} \\

  The files that are downloaded must correspond to the \destdir
  directory as produced when the component was actually built.  The
  \destdir directory of the component must be filled with the
  downloaded files.

\item[Helper Files] \mbox{} \\

  The program can write helper files to the build directory, but it
  must not use the build directory for intermediate files.

\item[Revision Control] \mbox{} \\

  The program may access the revision control system to determine if
  the component sources have changed.  It must not change the state of
  the revision control system -- switching branches, altering files,
  etc. -- due to the parallelism implicit in building with \lmsbw.

\item[Extracted File Ownership] \mbox{} \\

  Files extracted from the downloaded build must have their ownership
  set to the current user, and to a group that includes the current
  user.

\item[Extracted File Times] \mbox{} \\

  Files extracted from the downloaded build must have their time stamp
  set to the current time so that \make will function properly.

\end{description}

Upon a successful download, \lmsbw will time stamp all the downloaded
files with the current time stamp; this will ensure that any dependent
components will build as if the prerequisite component is entirely
new.

\subsection{Return Values}
The program that downloads build output must return one of the
values shown in the following table.  If any other value is returned,
\lmsbw treats it as an indeterminate fatal error.


\begin{description}
\item[0]{Successful download}

  All files that must be installed into the component's \destdir have
  been successfully downloaded \& installed.

\item[1]{Failure; sources have been locally modified}

  If the source has been locally modified, the script should report
  the situation by returning one (1).  The reasoning is that local
  source modifications are interpreted as an intention to build the
  component from source rather than downloading the build output.

  Since the component is set to download the build output, the local
  modifications would be silently ignored.  \lmsbw does not
  automatically build this component when this result is encountered
  because that would affect all dependent components, and their build
  rules have already be generated and provided to \make.

  The solution is to use the \texttt{--disable-build-output-download}
  option of \lmsbwcmd; see \xref{usinglmsbw:disable-build-output-download}.

\item[2]{Failure; no build corresponds to the sources}

  This failure indicates that the source present in the source
  directory has no corresponding build that can be downloaded.

  This is similar to the result value one (1), but this result is not
  intended to be used when local changes have been made to the
  sources.

  This means that the source corresponds to officially committed code,
  but the build machines do not have a build for it.

  Returning this value is strongly indicative of an internal
  infrastructure issue: the build machines are out-of-sync with the
  publicly available sources.  When this occurs, all builds will grind
  to a halt unless the developer explicitly causes the affected
  component to be built instead of being downloaded.

\item[3]{Missing command line argument}

  If the script is invoked without all the required parameters, it
  must return this value.

  The script may output text to \texttt{stderr} or \texttt{stdout}
  that indicates the missing command line parameter.

\end{description}

\section{Sample}

The \texttt{build-output-download} sample provides a complete example
of how to implement downloadable components.  See the \texttt{README}
in the sample directory for explicit instructions.

Also, refer to \xref{tips:using-build-output-download} for a step-by-step
guide on implementing such a system for local use.

%% \section{Some notes on implementation}

%% Download script must differentiate between a perfect match for a
%% downloaded build, and local changed.

%% Consider perforce change numbers.  Even if sources are changed, a
%% perforce number can be deteremined.  If sources are locally changed,
%% download script must not download new build output, unless there is a
%% perfect match.

%% \begin{enumerate}
%%   \item Source is still cloned into build directory.

%%   \item Download script must determine if source is different than the
%%     supplied download.  If different, it must fail.

%% \item command line argument processing

%%   If 'download', or 'no-download' specified on command line, check that
%%   download script defined.  Iterate for each component, setting
%%   attribute according to 'download' and 'no download' lists.

%% \item Generating Rules

%%   \begin{description}
%%   \item{download}

%%     Build output download components have additional work.

%%     \begin{itemize}
%%     \item For the first build, invoke download script
%%     \item if sources changed, invoke download script.
%%     \end{itemize}

%%     \item{no download}

%%       Components which do not have this attribute set to 'download'
%%       will generate rules as done already.
%%   \end{description}
%% \end{enumerate}

%% After the download has been completed, \lmsbw proper will be used to
%% install the software into the install directory.  Then, the source
%% manifest will be updated and the download request will never be
%% invoked until the source changes -- either manually or through a sync.

%% \todo{document that this affects the hash (\xref{variables:hash})}

%% All direct \& indirect prerequisites of a download component must also
%% be download.  This forces APIs to match.
