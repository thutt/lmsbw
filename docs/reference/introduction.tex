% Copyright (c) 2012 Taylor Hutt, Logic Magicians Software
%
% This program is free software: you can redistribute it and/or
% modify it under the terms of the GNU General Public License as
% published by the Free Software Foundation, either version 3 of the
% License, or (at your option) any later version.
%
% This program is distributed in the hope that it will be useful, but
% WITHOUT ANY WARRANTY; without even the implied warranty of
% MERCHANTABILITY or FITNESS FOR A PARTICULAR PURPOSE.  See the GNU
% General Public License for more details.
%
% You should have received a copy of the GNU General Public License
% along with this program.  If not, see <http://www.gnu.org/licenses/>.
%
\chapter{Introduction}

Recursive use of \make has fallen out of favor after the publication
of \emph{Recursive Make Considered Harmful}
(\texttt{http://aegis.sourceforge.net/auug97.pdf}), but in real-world
situations, it's still dominant because \make provides little support
towards actually creating a maintainable non-recursive build system.
The biggest stumbling block to implementing a non-recursive using
\make is the complete lack of scoping; symbol scoping is accomplished by
traditional recursive use of \make, and is all but non-existent in a
non-recursive implementation.  This is creates a maintenance issue
because the developers have to be very careful to end up with no
macro, target or implicit rule collisions.

Rather than trying to change the world by convincing people to
increase the complexity of their build system with non-recursive use
of \make, the software described in this document aims to simplify the
world by implementing a wrapper system -- implemented entirely in
\gnumake and \texttt{bash} scripts -- which allows you to continue to
use a recursive \make system, but provides you with dramatic
performance improvements by only recursing when something in the
source tree has changed and without requiring you to make widespread
changes to your existing build process.

In short, if your \nullbuild takes more time than the \mtree utility
takes to execute on all your source directories, then you can benefit
from \lmsbw.

The following tenets guided the implementation of \lmsbw:

\begin{itemize}
\item Incremental Builds

  \lmsbw was developed with a team of developers in mind and from that
  perspective the day of a developer involves many small changes and
  rebuilds of the software.  The key focus of \lmsbw is to ensure that
  incremental builds are as fast as possible.

  To achieve such speed, \lmsbw never recurses into a subdirectory if
  nothing in the corresponding source tree has changed.  This is
  accomplished by using the \mtree utility.  \todo{describe what mtree
    is.}

  A logical conclusion from this is this: if your current \nullbuild
  takes more than the total time of executing \mtree on each source
  directory, you can benefit from \lmsbw.

\item Disk Space

  Disk space is free relative to the cost of a developer.  The amount
  of time wasted when developers have to \texttt{make clean} because
  tool options have changed is a unnecessary, particularly if changing
  those options are to make a \emph{debug} buid, or to perform some
  special one-off test.

  \lmsbw saves time by compromising on space.  Because developers
  rarely need to know where their intermediate build output resides --
  a report is provided which shows where all files are placed, if you
  \emph{need to know} -- \lmsbw will place build output into
  directories based on many global options.  For example, \lmsbw can
  be configured to include the revision control system's branch name
  when determining the build output directories; this means that
  builds, regardless the revision control branch being used, would
  always be incremental.

  Trading off increased disk space with improved developer
  productivity is definitely the right choice if you've got market
  constraints.\todo{I do not like this paragraph.}

\item Source Is Immutable

  \lmsbw is a general wrapper for build systems, and generally
  speaking, each build systems perform differently: some write all
  output data into the source directory, while others write to a
  user-specified directory.

  Partially due to the unknown behavior of wrapped builds, \lmsbw has
  been designed to consider the source directory tree to be immutable.
  Further, to reduce the complexity of the implementation of LMSBW as
  well as to guarantee correctness, \lmsbw has been written to clone
  the original source directory into the build directory.

  The upshot of this is that the original source directory will never
  be sullied by \lmsbw, nor your own wrapped build process.  Another
  benefit of this is that you can easily use different toolchains --
  perhaps for cross compiling to another operating system or
  architecture -- without having to retool your own build process.

\item Uniform Interface

  \lmsbw has been designed to make it easy to wrap an existing build
  process, and to access portions of the wrapped build.  In concept,
  you will declare each one of your \emph{components} -- usually a
  source directory containing a \texttt{Makefile}.  Declaration
  includes information about the directory and options for building
  it, including which other component are prerequisites.

  \lmsbw compiles this information and generates, on-the-fly, a set of
  \make rules and targets which allow your wrapped build process to be
  executed.

  Included in this is a uniform interface: if you want to produce a
  build report for all components, you simply execute \texttt{lmsbw
    report}, but if you want to specifically get a report on the
  \emph{dailyprocessing} component, you would execute \texttt{lmsbw
    report.dailyprocessing}.

  Similarly, all other \emph{verbs} which are exported by \lmsbw can
  be executed globally or on an individual component.

  With the designed-in extensibility of \lmsbw, it's also quite simple
  to add your own verbs to the build process.

\item Extensible

  \lmsbw may not have all the features you need, but it uses the
  \emph{Gnu Make Standard Library} (gmsl), it's fully documented and
  it's easy to add new features.

\end{itemize}

\section{Will \lmsbw Benefit You?}

At this point you may be interested, but you're reticent to invest a
lot of time just to find out if \lmsbw will benefit your build.
Fortunately, there is a way to quickly determine if \lmsbw will
provide a benefit, because this source package has a handy script
which will provide you with benchmark numbers.  Just follow these
instructions.

\begin{enumerate}
\item Download \& extract \lmsbw sources

  Download and extract the sources.  We'll use an environment variable
  to note the location.

\begin{verbatim}
cd <directory where you extracted lmsbw>
ld=$(pwd)
\end{verbatim}
\item Build mtree utility
\begin{verbatim}
${ld}/scripts/lmsbw                                          \
   --configuration ${ld}/samples/hello-world/hello-world.cfg \
   --built-root /tmp/lmsbw-test  -- mtree
\end{verbatim}

The command above will output a message indicating where the \mtree
utility was placed; we'll use that in the next command.

\item Test Speed

  The following command will execute the \mtree utility on each
  directory named on the command line, and then output the elapsed
  time for processing all the directories.

\begin{verbatim}
  ${ld}/scripts/lmsbw-mtree-speed-test         \
     --mtree <full path output by step above>  \
     <source directory>...
\end{verbatim}

To get a representative sample of how long a normal \nullbuild will
take for your project, you should run this command multiple times to
ensure that the in-RAM disk caches are filled.

\end{enumerate}

In short, the \mtree utility performs a \texttt{stat} on every file in
the named directories and produces output which can be used later to
determine if any file in the source tree has changed.  By knowing that
no files have changed in a source directory, \lmsbw will never recurse
into that directory to build -- modulo other dependency issues which
can force a re-build.\todo{This paragraph should be placed where mtree
  is first mentioned.}

What this means, for you, is that the elapsed time shown is a very
good approximation for how long a full \nullbuild for your project
will take.  Developers performing incremental builds will greatly
benefit by never recursively building source directories that have not
changed.

\section{Steps To Wrap Your Build}

The work needed to wrap your build with \lmsbw is directly
proportional to the size of your build process; a small project will
obviously take less time than a large project, but in both cases the
steps are the same.

\subsection{Configuration File}

You must provide a configuration file to \lmsbw.  The main purpose of
the configuration file is to create and set up the \lmsbwconfiguration
variable.

Primarily \lmsbwconfigurationfield{load-configuration-function} is set
to the name of a Gnu \make function which is responsible for setting
up the configuration for each component that will be included in your
build.

Other fields of \lmsbwconfiguration can be optionally set to further
control the behavior of \lmsbw.

\subsection{Loading Component Configuration}

After loading the configuration file, \lmsbw will invoke the
configuration-loading function.  This function is responsible for
defining all the components that are to be built.  Generally there is
a one-to-one correspondence of components to component configuration
files, which you provide, and the called function will load each of
these files using \gnumake \texttt{\$(include)}
directive\todo{Describe this is shown in samples?}.

The configuration files are, in fact, \makefile portions which can
have rules that will override the default behavior of \lmsbw.  For
example, each component's build system is supposed to have a
\texttt{install} target, but sometimes its not always possible to
modify the component's build process.  In this case, as the writer of
the component configuration file, you can override the normal \lmsbw
behavior of recursively invoking \texttt{make install} on the
component by providing your own \emph{install} rule in the component's
configuration file\todo{Have a sample with override rules}.

This configuration-loading function must ultimately produce two sets
of data:

\begin{itemize}
\item \lmsbwcomponents

  This is a standard \gnumake variable which is a list of all
  the component names which are to be built.

\item \lmsbwcomponent{<component-name>}

  Each component loaded is assigned a \gmsl associative array to
  hold all the data relevent to building the software.

  The keys that are defined for the component's associative array are
  dependent upon the \emph{type} of the component.  For example, a
  \emph{source} component will define a set of keys which would
  include information such as the source directorry.  A
  \emph{download} component would have different keys, such as the
  \texttt{URL} from which the compoenent can be downloaded.

  This flexibility allows new component types to be created without
  affecting existing component types.
\end{itemize}


\subsection{\lmsbwcomponent{<component-name>} Configuration}

Each component's associative array contains various bits of
user-provided data which enables \lmsbw to build and install it.

Such user-provided data, for a \emph{source} component, includes:

\begin{itemize}
\item source directory
\item prerequisite components
\item brief description
\item reason for building
  \begin{itemize}
  \item built to be delivered in the product
  \item built to be used in the build
  \end{itemize}
\end{itemize}

Conveniently, there are \lmsbw functions, such as
\texttt{declare\_source\_component}, which simplify the configuration
of components.

\subsection{Recursively Invoking \make}

During the execution of the build process, \lmsbw recursively runs
three different \makefile processes:

\begin{itemize}
\item \lmsbw Makefile

  The first is the main \lmsbw \makefile which handles all the
  configuration and interdependencies between high-level components.

\item Component Wrapper \makefile

  The second is a subserviant \makefile, yet still part of \lmsbw,
  which performs component-specific configuration such as configuring
  standard \makefile macros like \texttt{CC} and \texttt{LD} to
  correspond to the toolchain which will be used to actually build the
  component.

  It also ensures that each component's build process is completely
  disjoint from other components; the settings for \emph{component a}
  cannot interfere with the settings for \emph{component b}.

\item Component \makefile

  Finally, the \makefile of the component is executed.  This will
  likely already exist if you are wrapping an existing build process.

  This \makefile will be entered at most twice for each build.  The
  first invocation will be to \emph{build} the component, and the
  second time will be to \emph{install} the component.
\end{itemize}
