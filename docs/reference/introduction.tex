% Copyright (c) 2012 Taylor Hutt, Logic Magicians Software
%
% This program is free software: you can redistribute it and/or
% modify it under the terms of the GNU General Public License as
% published by the Free Software Foundation, either version 3 of the
% License, or (at your option) any later version.
%
% This program is distributed in the hope that it will be useful, but
% WITHOUT ANY WARRANTY; without even the implied warranty of
% MERCHANTABILITY or FITNESS FOR A PARTICULAR PURPOSE.  See the GNU
% General Public License for more details.
%
% You should have received a copy of the GNU General Public License
% along with this program.  If not, see <http://www.gnu.org/licenses/>.
%
\chapter{Introduction}

Recursive use of \make has fallen out of favor since the publication
of \emph{Recursive Make Considered Harmful}
(\texttt{http://aegis.sourceforge.net/auug97.pdf}), but in real-world
situations, recursive \make is still dominant because \make provides
no support for creating a maintainable non-recursive build system.

Rather than trying to change the world by convincing people to
increase the complexity of their build system with non-recursive use
of \make, the software described in this document aims to simplify the
world by implementing a wrapper system -- implemented entirely in
\gnumake and \texttt{bash} scripts -- that allows the continued use of
a recursive \make system, but provides dramatic performance
improvements by only recursing when something in the source tree has
changed and without requiring widespread changes to the existing build
process.

The following tenets guided the implementation of \lmsbw:

\begin{itemize}
\item Incremental Builds

  \lmsbw was developed with a team of developers in mind, and from that
  perspective the day of a developer involves many small changes and
  rebuilds of the software.  The key focus of \lmsbw is to ensure that
  incremental builds are as fast as possible.

  To achieve such speed, in simple terms, \lmsbw will never recurse
  into a sub-directory unless something in the corresponding source
  tree has changed.  This is accomplished by using the \mtree utility.

\item Disk Space

  Disk space is free relative to the cost of a developer: the amount
  of time wasted by developers performing administrative tasks around
  their build \& build output is a giant productivity drain.

  \lmsbw saves time by compromising on space.  Trading off increased
  disk space for improved developer productivity is the right choice
  to reduce product development cost.

\item Source Is Immutable

  \lmsbw is a general wrapper for build systems, and generally
  speaking, each build system performs differently: some write all
  output data into the source directory, while others write to a
  user-specified build directory.

  Partially due to the unknown behavior of wrapped builds, \lmsbw has
  been designed to consider the source directory tree to be immutable.
  Further, to reduce the complexity of the implementation of \lmsbw,
  as well as to guarantee correctness, \lmsbw has been written to
  clone the original source directory into the build directory.

  The upshot of this is that the original source directory will never
  be sullied by \lmsbw, nor the wrapped build process.  Another
  benefit of having an immutable source tree is that it is easy to use
  different toolchains -- perhaps for cross compiling to another
  operating system or architecture -- without having to retool the
  original build process.

\item Uniform Interface

  \lmsbw has been designed to make it easy to wrap an existing build
  process.  In concept, declare each \emph{component}, usually a
  source directory containing a \makefile, to \lmsbw.  Declaration
  includes information about the directory and options for building
  it, including which other components are prerequisites.

  \lmsbw compiles this information and generates, on-the-fly, a set of
  \make rules and targets that allows the wrapped build process to be
  executed.

  Included in the generation of rules and targets is a uniform
  interface: to produce a build report for all components, just
  execute the \texttt{report} verb (\xref{usinglmsbw:report}), but it
  can also be executed on a particular component with this syntax:

  \begin{center}
    \texttt{report.dailyprocessing}
  \end{center}

  Similarly, many other \emph{verbs} (\xref{usinglmsbw:target:verbs})
  that are exported by \lmsbw can be executed globally or on an
  individual component.

\item Extensible

  \lmsbw has been designed to be easily extended with new types of
  components.  See \xref{chap:extending}.
\end{itemize}

\section{What Benefit Will \lmsbw Provide?}

At this point there may be interest, but also reticence, to invest a
lot of time just to find out if \lmsbw will provide a speed benefit.
Fortunately, there is a way to quickly determine if \lmsbw will
provide a benefit, because this package has a handy script that will
provide benchmark numbers.  Just follow these instructions:

\begin{itemize}
\item Download \& extract \lmsbw sources

\item Set environment variable

  Set \texttt{ld} to the full pathname of the directory to which
  \lmsbw was extracted.

\item Test Speed

\begin{verbatim}
  ${ld}/scripts/lmsbw-mtree-speed-test   \
     <source directory>...
\end{verbatim}

  The command will execute the \mtree utility on each directory named
  on the command line, and then output the elapsed time for processing
  all the directories.

\end{itemize}

\begin{center}\framebox{\begin{minipage}{.9\linewidth}\textbf{NOTE}

This test will create \texttt{/tmp/lmsbw-test} to hold the
\texttt{mtree} utility.  After testing is complete,
\texttt{/tmp/lmsbw-test} can be deleted.

\end{minipage}}\end{center}


The output of this test is a rough estimate of what the \nullbuild
time should be; to get a better estimate of how long a normal
\nullbuild will take, run this command multiple times to ensure that
the in-RAM disk caches are filled.

In short, the \mtree utility performs a \texttt{stat} on every file in
the named directories and produces output that can be used later to
determine if any file in the directory tree has changed.  When no
files have changed in a source directory, \lmsbw will not recurse into
that directory to build -- modulo other dependency issues that can
force the component's build to be re-invoked.

This means that the elapsed time shown is a very good approximation
for how long a full \nullbuild of the project will take.  Developers
performing incremental builds will greatly benefit by never
recursively building source directories that have not changed.

To see how well \lmsbw fares on a sample build system with tens of
thousands of source files, please see \xref{samples:generate}.

To see the relationship between \emph{number of component
  configuration files} and \emph{\lmsbw startup time}, please refer to
\xref{samples:startup-time}.

\section{How \lmsbw Works}

\lmsbw is not a system that actually builds software; rather, it is a
wrapper for an existing build system that, in most cases, will improve
the prompt-to-prompt incremental build performance of the system in
use now.  \lmsbw can improve an existing system because many build
processes perform altogether too much work when none needs to be
performed; \lmsbw cuts out this unnecessary work by never invoking the
wrapped build when nothing has changed since the last build.

Let's examine the what makes \lmsbw different:

\begin{description}
\item{Defined Build Process API}

  To facilitate rebuilding dependent components when necessary --
  especially when the exported API of a prerequisite component changes
  -- \lmsbw, itself, defines an API that wrapped builds must follow.
  This API requires that each component install its public build
  product into a \texttt{DESTDIR}.  This facilitates automatic
  checking of inter-component APIs that will force dependent
  components to be rebuilt upon changes to the API.

\item{Automatic Rule Generation}

  \lmsbw automatically generates the rules that are used by \make to
  build the project, but to be able to generate these rules, every
  component in the project must be declared to \lmsbw.

  Once declared, \lmsbw takes control of all aspects of each
  component's build: the build directory, the \texttt{DESTDIR}, the
  install directory, and the toolchain to use.  When the component's
  build process is invoked, it only needs to focus on producing a
  build, and, when requested, installing it into the \texttt{DESTDIR}
  directory.

\item{Recurse only when needed}\label{intro:build-component-criteria}

  \lmsbw maintains the inter-component dependencies that are
  explicitly defined by the component declarations and the
  intra-component file dependencies are handled by each component's
  build process.  Not handling intra-component file dependencies
  reduces memory use and makes the criteria for recursing easier to
  determine, thereby simplify \lmsbw.

  The general determination for recursing into each component's build
  process is this:

  \begin{itemize}
  \item{If the component is not built}
  \item{If \emph{any} file in the source tree has been changed}
  \item{If \emph{any} file in a prerequisite's declared API has changed}
  \item{If the component's configuration file has changed}
  \end{itemize}

  Any of the above conditions being true causes \lmsbw to recurse into
  the component's build process.  Conversely, if \emph{all} of the
  conditions are false, then \lmsbw will \emph{not} recurse into the
  component's build process; not recursing for components that are
  already built saves an amount of time that is proportional to the
  number and complexity of \make rules present in that directory.

\item{\makefile Simplicity}

  During the execution of the build process, \lmsbw executes, at most,
  three different \makefile processes:

  \begin{itemize}
  \item{\lmsbw \makefile}

    The first is the main \lmsbw \makefile tbat handles all the
    configuration and dependencies between components.

  \item{Component Trampoline \makefile}

    The second \makefile performs configuration such as standard
    macros for the toolchain (\xref{chap:toolchain-usage})
    selected to build the component.

    It also ensures that each component's build process is completely
    disjoint from other components; the settings for one component
    cannot interfere with the settings for any other component.

    Because each individual component is fulfilled through a separate
    \make process, this trampoline also ensures that the parallelism
    available through \gnumake is effectively used when it becomes
    necessary to invoke a component's build process.

  \item{Component \makefile}

    Finally, the \makefile of the component is executed.  This will
    likely already exist if wrapping an existing build process.

    This \makefile will be entered at most once for each build.  The
    invocation will be to \emph{install} the component; the \makefile
    itself must have appropriate rules configured so that the
    component will be built each time it is entered.
\end{itemize}

\item{Share Build Output}

  If the project contains different SKUs, and those SKUs share
  components, it is possible for \lmsbw to share the build output from
  one SKU with another, if the components are compiled in exactly the
  same way.

\end{description}

\lmsbw is a lightweight system that has been designed for speeding up
incremental builds.  It handily achieves that goal by performing as
little work as possible from run-to-run.  See
\xref{samples:startup-time} for examples of startup speed and memory
usage.

\section{Steps To Wrap A Build}

The work needed to wrap a build with \lmsbw is directly proportional
to the size of the original build process; a small project will
obviously take less time than a large project, but in both cases the
steps are the same.

\subsection{Master Configuration File}

A master configuration file for the project must be provided to \lmsbw
on the command line.  The main purpose of the file is to set up the
\lmsbwconfiguration variable~ (\xref{variables:lmsbw-configuration}).

A \gnumake function that will load the component configuration files
(\xref{variables:load-configuration-function}), and an indication of
what \emph{kind} (\xref{variables:kind}) of components the project
will use (\xref{variables:component-build-support}) both must be
provided.

Other fields of \lmsbwconfiguration can be optionally set to further
control the behavior of \lmsbw.
See~\xref{variables:lmsbw-configuration}.

\subsection{Loading Component Configuration}

After loading the command-line-specified master configuration file,
\lmsbw will invoke the component configuration loading function
(\xref{variables:load-configuration-function}).  This function is
responsible for declaring all the components that are to be included
in the project.  Generally there is a one-to-one correspondence of
components to component configuration files, but this is not a
requirement.  The called function will generally load each component
configuration file using the \gnumake \texttt{\$(include)} directive.

The component configuration files are, in fact, \makefile snippets
that can actually have rules that will override the default behavior
of \lmsbw.  See \xref{chap:overriding} for details on overriding the
default behavior of component building.

This configuration-loading function must ultimately produce two sets
of data:

\begin{itemize}
\item \lmsbwcomponents

  This is a standard \gnumake variable that is a list of all the
  component names that are to be built.

  This variable is internally managed as a by-product of declaring
  components when using the standard declaration functions.

\item \lmsbwcomponent{component}

  Each component contained in \lmsbwcomponents is assigned an
  associative array that holds all the data relevant to building the
  component.

  The keys that are defined for the component's associative array are
  dependent upon the \emph{kind} (\xref{variables:kind}) of the
  component.  For example, a \emph{source} component will define a set
  of keys that would include information such as the source directory.
  A \emph{download}\footnote{Hypothetical case; a pure
    \texttt{download} component type does not exist.} component would
  have different keys, such as the \texttt{URL} from which the
  component can be downloaded.
\end{itemize}


\subsubsection{\lmsbwcomponent{component} Configuration}

Each component's associative array contains data that enables \lmsbw
to build and install it.

Such user-provided data, for a \emph{source} component, includes:

\begin{itemize}
\item source directory
\item prerequisite components
\item brief description
\item reason for building
  \begin{itemize}
  \item built to be delivered in the product
  \item built to be used in the build
  \end{itemize}
\end{itemize}

Conveniently, there are \lmsbw functions, such as
\texttt{declare\_source\_component}
(\xref{api:declare-source-component}), that simplify the configuration
of components.
