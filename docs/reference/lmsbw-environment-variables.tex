\chapter{Environment Variables}

\lmsbw provides several variables to facilitate ease-of-use.  This
chapter describes the variables that are intended to be used by
developers using this system.

\section{Entire \lmsbw Process}

Variables listed in the following sections are exported by \lmsbwcmd
and are available to the main \lmsbw process, and each component build
process.

\subsection{\texttt{LMSBW\_ADAPTER\_SCRIPT}}\label{envvar:lmsbw-adapter-script}

This variable holds the full pathname of a program that will be
automatically used to adapt the output of \lmsbw to the build output
structure of a different build process.

This will be typically used during the time that a build process is
transitioned to \lmsbw.

See \xref{api:declare-adapter-script}.

The script referenced by this variable will be invoked with a single
command line argument that is the full pathname to a file describing
each of the components that potentially needs to be adapted to the old
build process.

The format of the file is:

\begin{verbatim}
component-file := (component-info '\n')...
component-info := component-name ':' build-root-directory
\end{verbatim}

The \texttt{component-name} is the name of the component as declared
to \lmsbw in the configuration file.

The \texttt{build-root-directory} is the full path to the root of the
build directory for the named component.  The
\texttt{build-root-directory} contains the \texttt{build} and
\texttt{destdir} directories.

The \texttt{destdir} directory contains the exported build output that
was installed by the component's build process.  It is from this
directory that the component's build output should be copied to the
old build system's build directory.

Since the adapter script is run unconditionally for each invocation of
\lmsbw, it can write sentinel files in the
\texttt{build-root-directory} to reduce the number of times the build
files must be copied.

The adapter script can also launch the old build process, though
preferably it should only launch the old build process if something
has actually changed.

The \texttt{adapter} sample (\xref{samples:adapter}) shows how
this can be done.

\subsection{\texttt{LMSBW\_VERBOSE}}

This variable is created when \texttt{--verbose}
(\xref{usinglmsbw:verbose}) is used with \lmsbwcmd.

It is used to control verbosity within \lmsbw.

\subsection{\texttt{LMSBW\_BUILD\_ROOT}}

This variable is assigned the value of the \texttt{--build-root}
(\xref{usinglmsbw:build-root}) parameter.

\subsection{\texttt{LMSBW\_CONFIGURATION\_FILE}}

This variable is assigned the value of the \texttt{--configuration}
(\xref{usinglmsbw:build-root}) parameter.

\subsection{\texttt{LMSBW\_DISABLE\_BUILD\_OUTPUT\_DOWNLOAD}}

This variable is created when \texttt{--disable-build-output-download}
(\xref{usinglmsbw:disable-build-output-download}) is used with
\lmsbwcmd.

It is used to disable the downloading of build output for components
appropriately attributed (\xref{variables:build-output-download}).  In
other words, it's used to build components that would normally have
their build output downloaded.

\subsection{\texttt{LMSBW\_BUILD\_OUTPUT\_NO\_DOWNLOAD}}

This variable contains the list of components named with
\texttt{--no-download} (\xref{usinglmsbw:no-download}) parameter.

This is used internally to cause components, marked to have their
build output download, to be actually built.

\subsection{\texttt{LMSBW\_PARALLEL\_LEVEL}}

This variable is assigned the value of the \texttt{--parallel}
(\xref{usinglmsbw:parallel}) parameter.

It is used to control the number of parallel invocations of component
build processes.

\subsection{\texttt{LMSBW\_PREREQUISITE\_CHECK\_COMPONENT}}

This variable is assigned the value of the
\texttt{--prerequisite-check} (\xref{usinglmsbw:prerequisite-check})
parameter.

It is used to ensure that an individual component has configured the
proper set of prerequisite components.

\subsection{\texttt{LMSBW\_PROGRESS}}

This variable is created when \texttt{--progress}
(\xref{usinglmsbw:progress}) is used with \lmsbwcmd.

When this is set, \lmsbw will produce more \emph{progress}-related
output during a build.

\subsection{\texttt{LMSBW\_ELAPSED\_TIME}}

This variable is created when \texttt{--time} (\xref{usinglmsbw:time})
is used with \lmsbwcmd.

When set, \lmsbw will output the elapsed time to build each component.

\subsection{\texttt{LMSBW\_TOOLCHAIN}}

This variable is assigned the value of the \texttt{--toolchain}
(\xref{usinglmsbw:toolchain}) parameter.

This is used internally to use a specific toolchain when building all
components.

\subsection{\texttt{LMSBW\_TOOLCHAINS\_ROOT}}
\label{envvar:lmsbw-toolchains-root}

This variable is assigned the value of the \texttt{--toolchains-root}
(\xref{usinglmsbw:toolchains-root}) parameter.

This is used internally to use a specific toolchain when building.

\subsection{\texttt{LMSBW\_SCRIPT\_DIRECTORY}}
\label{envvars:lmsbw-script-directory}
Since \lmsbw need not be put in the path, it is important to be able
to execute any script present in the \lmsbw system.

\texttt{LMSBW\_SCRIPT\_DIRECTORY} is the full pathname of the
directory containing the scripts associated with the \lmsbw package.

This variable is exported as a convenience, and should be used as
convention, to directly access scripts; do not assume that
\lmsbw-based scripts are in the path.

\subsection{\texttt{GMSL}}

This is the absolute pathname of the directory containing the \gmsl
library.  To use \gmsl in a component's build process, do the
following:

\begin{verbatim}
  include $(GMSL)/gmsl
\end{verbatim}

\section{Component Build Process}

Variables listed in the following sections are exported by \lmsbw and
are only available to component build processes.

\subsection{\destdir}

The \destdir variable must be used as a root-directory prefix on the
component \emph{install} rules.

\lmsbw creates the directory and sets \destdir to the absolute
pathname.  The rules may assume the directory exists and that full
control of the directory is available.


\subsection{\texttt{LMSBW\_C\_TOOLCHAIN}}\label{wrap:lmsbw-c-toolchain}

If a component has a toolchain associated with it, this variable will
be set to the toolchain directory name.  If there is no aasociated
toolchain, this variable will not be set.

\subsection{\texttt{LMSBW\_C\_TOOLCHAINS\_ROOT}}\label{wrap:lmsbw-c-toolchains-root}

This will be set to \texttt{LMSBW\_TOOLCHAINS\_ROOT}
(\xref{envvar:lmsbw-toolchains-root}) if a component has a toolchain
associated with it.  If there is no aasociated toolchain, this
variable will not be set.

\subsection{\texttt{LMSBW\_C\_BUILD\_INSTALL\_DIRECTORY}}\label{wrap:build-install-directory}

Use this as a prefix to access the build output for components that
have their \texttt{reason} (\xref{variables:reason}) variable set to
\texttt{build}.

See \xref{wrap:using-build-components} for details on using this variable.

\subsection{\texttt{LMSBW\_C\_INSTALL\_DIRECTORY}}

This is the root directory where all components are installed.  Use
this as the base directory when accessing the API directories exported
by other, prerequisite, components.

\subsection{\texttt{LMSBW\_C\_BUILD\_DIRECTORY}}\label{wrap:lmsbw-c-build-directory}

This is set to the root of the build directory for the associated
component.  The component \makefile can place files such as \mtree
manifests or \make sentinels into this directory.

This is \emph{not} the directory in which the component \makefile is
started.   Do not put object files into this directory.

Compare \xref{wrap:lmsbw-c-build-working-directory}.

\subsection{\texttt{LMSBW\_C\_BUILD\_WORKING\_DIRECTORY}}\label{wrap:lmsbw-c-build-working-directory}

This variable holds the actual directory in which the component's
build will be executed\footnote{The \texttt{-C} argument of \make}.

Compare \xref{wrap:lmsbw-c-build-directory}.

