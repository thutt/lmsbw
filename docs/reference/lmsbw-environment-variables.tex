\chapter{Environment Variables}

\lmsbw provides several variables to facilitate ease-of-use.  This
chapter describes the variables which are intended to be used by
developers using this system.

\section{Entire \lmsbw Process}

Variables listed in the following sections are exported by \lmsbwcmd
and are available to the main \lmsbw process, and each component build
process.

\subsection{\texttt{LMSBW\_VERBOSE}}

This variable is created when \texttt{--verbose}
(\xref{usinglmsbw:verbose}) is used with \lmsbwcmd.

It is used to control verbosity within \lmsbw.

\subsection{\texttt{LMSBW\_BUILD\_ROOT}}

This variable is assigned the value of the \texttt{--build-root}
(\xref{usinglmsbw:build-root}) parameter.

\subsection{\texttt{LMSBW\_CONFIGURATION\_FILE}}

This variable is assigned the value of the \texttt{--configuration}
(\xref{usinglmsbw:build-root}) parameter.

\subsection{\texttt{LMSBW\_DISABLE\_BUILD\_OUTPUT\_DOWNLOAD}}

This variable is created when \texttt{--disable-build-output-download}
(\xref{usinglmsbw:disable-build-output-download}) is used with
\lmsbwcmd.

It is used to disable the downloading the build output for components
marked to download the build output.  In other words, it's used to
build components which would normally have their build output
downloaded.

\subsection{\texttt{LMSBW\_BUILD\_OUTPUT\_NO\_DOWNLOAD}}

This variable contains the list of components named with
\texttt{--no-download} (\xref{usinglmsbw:no-download}) parameter.

This is used internally to cause components, marked to have their
build output download, to be actually built.

\subsection{\texttt{LMSBW\_PARALLEL\_LEVEL}}

This variable is assigned the value of the \texttt{--parallel}
(\xref{usinglmsbw:parallel}) parameter.

It is used to control the number of parallel invocations of component
build processes.

\subsection{\texttt{LMSBW\_PREREQUISITE\_CHECK\_COMPONENT}}

This variable is assigned the value of the
\texttt{--prerequisite-check} (\xref{usinglmsbw:prerequisite-check})
parameter.

It is used to ensure that an individual component has configured the
proper set of prerequisite compoenents.

\subsection{\texttt{LMSBW\_PROGRESS}}

This variable is created when \texttt{--progress}
(\xref{usinglmsbw:progress}) is used with \lmsbwcmd.

When this is set, \lmsbw will produce more \emph{progress}-related
output during a build.

\subsection{\texttt{LMSBW\_ELAPSED\_TIME}}

This variable is created when \texttt{--time} (\xref{usinglmsbw:time})
is used with \lmsbwcmd.

When set, \lmsbw will output the elapsed time to build each component.

\subsection{\texttt{LMSBW\_TOOLCHAIN}}

This variable is assigned the value of the \texttt{--toolchain}
(\xref{usinglmsbw:toolchain}) parameter.

This is used internally to use a specific toolchain when building all
components.

\subsection{\texttt{LMSBW\_TOOLCHAINS\_ROOT}}

This variable is assigned the value of the \texttt{--toolchain-root}
(\xref{usinglmsbw:toolchain-root}) parameter.

This is used internally to use a specific toolchain when building.

\subsection{\texttt{LMSBW\_SCRIPT\_DIRECTORY}}

Since \lmsbw need not be put in the path, it is important to be able
to execute any script present in the \lmsbw system.

\texttt{LMSBW\_SCRIPT\_DIRECTORY} is the full pathname of the
directory containing the scripts associated with the \lmsbw package.

This variable is exported as a convenience, and should be used as
convention, to directly access scripts; do not assume that
\lmsbw-based scripts are in the path.

\subsection{\texttt{GMSL}}

This is the absolute pathname of the directory containing the \gmsl
library.  To use \gmsl in a component's build process, do the
following:

\begin{verbatim}
  include $(GMSL)/gmsl
\end{verbatim}

\section{Component Build Process}

Variables listed in the following sections are exported by \lmsbw and
are only available to component build processes.

\subsection{\destdir}

The \destdir variable must be used as a root-directory prefix on the
component \emph{install} rules.

\lmsbw creates the directory and sets \destdir to the absolute
pathname.  The rules may assume the directory exists and that full
control of the directory is available.


\subsection{\texttt{LMSBW\_C\_BUILD\_INSTALL\_DIRECTORY}}\label{wrap:build-install-directory}

Use this as a prefix to access the build output for components that
have their \texttt{reason} (\xref{variables:reason}) variable set to
\texttt{build}.

See \xref{wrap:using-build-components} for details on using this variable.

\subsection{\texttt{LMSBW\_C\_INSTALL\_DIRECTORY}}

This is the root directory where all components are installed.  Use
this as the base directory when accessing the API directories exported
by other, prerequisite, components.

\subsection{\texttt{LMSBW\_C\_BUILD\_DIRECTORY}}\label{wrap:lmsbw-c-build-directory}

This is set to the root of the build directory for the associated
component.  The component \makefile can place files such as \mtree
manifests or \make sentinels into this directory.

This is \emph{not} the directory in which the component \makefile is
started.   Do not put object files into this directory.

Compare \xref{wrap:lmsbw-c-build-working-directory}.

\subsection{\texttt{LMSBW\_C\_BUILD\_WORKING\_DIRECTORY}}\label{wrap:lmsbw-c-build-working-directory}

This variable holds the actual directory in which the component's
build will be executed.

Compare \xref{wrap:lmsbw-c-build-directory}.

\subsection{\texttt{LMSBW\_C\_CFLAGS}}

This is the union of the global \texttt{CFLAGS} value managed by
\lmsbw and the per-component value set via
\texttt{\texttt{component\_attribute\_cflags}} (\xref{api:cflags}).

The component \makefile should append \texttt{LMSBW\_C\_CFLAGS} to
\texttt{CFLAGS}.

