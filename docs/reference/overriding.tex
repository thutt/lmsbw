% Copyright (c) 2012 Taylor Hutt, Logic Magicians Software
%
% This program is free software: you can redistribute it and/or
% modify it under the terms of the GNU General Public License as
% published by the Free Software Foundation, either version 3 of the
% License, or (at your option) any later version.
%
% This program is distributed in the hope that it will be useful, but
% WITHOUT ANY WARRANTY; without even the implied warranty of
% MERCHANTABILITY or FITNESS FOR A PARTICULAR PURPOSE.  See the GNU
% General Public License for more details.
%
% You should have received a copy of the GNU General Public License
% along with this program.  If not, see <http://www.gnu.org/licenses/>.
%
\chapter{Overriding Default Rules} \label{chap:overriding}

Occasionally there will be a component \makefile that cannot be edited
for various reasons that does not conform to the \lmsbw requirements.

When \lmsbw's default for producing a component is insufficient or
when you cannot modify the component's build process, it is possible
to override the built-in method for building \& installing by
providing particularly named targets to the component's configuration
file.

This chapter describes the process to override these aspects of
producing a component.

\section{Overriding Build}\label{overriding:overriding-build}

In normal practice, it should be exceedingly rare for a component
build process to be overridden; even though it should be rare, \lmsbw
still provides a method of doing just that.  Here are a few of the
reasons why you might want to override the build phase of a component:

\begin{itemize}
\item Test \bni Infrastructure by not building a component

  How does your \bni system function if a required component is not
  actually built?

  Will it fail during \emph{install}, or while integrating into a
  production deliverable?  Will it fail with an easy-to-diagnose
  error?

  The simple test would be to add a build override target that simply
  does nothing.

\item Test \bni Infrastructure by corrupting build

  Will your \bni system recognize if a build has been corrupted?
  Perhaps only some files are built, or maybe the executables are
  corrupt, or the executables are for a different architecture?

  You could override the build target, build the component (using the
  \lmsbw infrastructure) and then corrupt the build output in-situ.

\item Apply a patch in-situ

  The component's source tree cannot be changed in the source tree,
  but you can apply a patch to the sources after they have been copied
  to the build directory.

  To do this, you'd override the build rule to apply the patches, then
  use the \lmsbw infrastructure to invoke the component's build
  process.

\item Component may not need a build; only installs static files

  Perhaps the component doesn't have a build; maybe only static files
  are installed.  In this case, you could override the build to do
  nothing, and save a minute amount of time.

\item Other unknown reasons
\end{itemize}

Once you have determined that you need to override the automatic rules
to perform a component build, the actual work is quite trivial; all
you need to do is add a specially named \makefile target to your
component configuration file, and supply the new commands to
associated with the target.  The name of the target is based on the
component name; if the name is \texttt{alpha}, you would provide the
following rule:

\begin{verbatim}
component.build.alpha:
        echo "Overriding trampoline; not building alpha";
\end{verbatim}

If your component configuration file does \emph{not} have this
specific target, then the default build rules of \lmsbw will take
effect, and your component's \makefile will be invoked to build the
component.

If you ultimately want to execute the component's actual build
process, you can use the same code of \lmsbw proper to initiate this
by executing:

\begin{footnotesize}
\begin{verbatim}
$(MAKE)                                                          \
    $(DEFAULT_$(call uc,$(LMSBW_C_KIND))_COMPONENT_MAKE_OPTIONS) \
    $(LMSBW_C_BUILD_TARGET)
\end{verbatim}
\end{footnotesize}

You must be aware when overriding default targets that the override
does not provide any prerequisites.  If you override the build target,
for example, without the proper prerequisites, you will not be able to
use the function shown above to build the component.  See the
\texttt{override} sample for details, and look at
\texttt{component.makefile} to determine the necessary prerequisites.

\section{Overriding Install}\label{overriding:overriding-install}

Overriding the default install rules is a task that will happen more
frequently than overriding the build rules.  Here are some reaons why
you might need to override the default install rules:

\begin{itemize}
\item Test \bni Infrastructure by not installing component
\item Component \makefile has no install rule, and cannot be modified
\item Component \makefile does not use \destdir
\item Component does not install all needed files
\item Other unknown reasons
\end{itemize}

Once you have determined the need to override the automatic rules to
perform a component install, the actual work is quite trivial; all you
need to do is add a \makefile target to your component configuration
file, and supply the new commands to associated with the target.  The
name of the target is based on the component name; if the component
name is \texttt{alpha}, you would do the following:


\begin{small}
\begin{verbatim}
component.install.alpha:
        $(MKDIR) --parents $(DESTDIR)/usr/include/alpha
        $(CP) --recursive                                  \
             $(LMSBW_C_BUILD_WORKING_DIRECTORY)/include/*  \
             $(DESTDIR)/usr/include/alpha
\end{verbatim}
\end{small}

If your component configuration file does \emph{not} have this
specific target, then the default install rules of \lmsbw will take
effect, and your component's \makefile will be invoked to install the
component.

If you ultimately want to execute the component's actual build
process, you can use the same code of \lmsbw proper to initiate this
by executing:

\begin{footnotesize}
\begin{verbatim}
$(MAKE)                                                          \
    $(DEFAULT_$(call uc,$(LMSBW_C_KIND))_COMPONENT_MAKE_OPTIONS) \
    $(LMSBW_C_INSTALL_TARGET)
\end{verbatim}
\end{footnotesize}


You must be aware when overriding default targets that the override
does not provide any prerequisites.  If you override the build target,
for example, without the proper prerequisites, you will not be able to
use the function shown above to build the component.  See the
\texttt{override} sample for details, and look at
\texttt{component.makefile} to see the determine the necessary
prerequisites.

