% Copyright (c) 2012 Taylor Hutt, Logic Magicians Software
%
% This program is free software: you can redistribute it and/or
% modify it under the terms of the GNU General Public License as
% published by the Free Software Foundation, either version 3 of the
% License, or (at your option) any later version.
%
% This program is distributed in the hope that it will be useful, but
% WITHOUT ANY WARRANTY; without even the implied warranty of
% MERCHANTABILITY or FITNESS FOR A PARTICULAR PURPOSE.  See the GNU
% General Public License for more details.
%
% You should have received a copy of the GNU General Public License
% along with this program.  If not, see <http://www.gnu.org/licenses/>.
%
\chapter{Overriding Default Rules via Configuration} \label{chap:overriding}

In some cases, you'll have a component \makefile that cannot be edited
for various reasons (office politics, code maintained elsewhere,
etc.), and it doesn't conform to the \lmsbw requirements.

When \lmsbw's default for producing a component is insufficient or
when you cannot modify the component's build process, it is possible
to override the built-in method for building \& installing.  \lmsbw
provides the ability to alter the behavior of both the \makefile
target for building and the target for installing the component by
adding particularly named targets to the component's configuration
file.

\todo{Note that it is possible to override \emph{other} components,
  too.  But, never do that!}

This chapter describes the process needed to override these apsects of
building a component.

\todo{Need a sample build which does both overrides}

\section{Overriding Build}

In normal practice, it should be exceedingly rare for a\todo{Need to
  put actual makefile invocation into a function which can be used.}
component's actual build process to be overriding; even though it
should be rare, \lmsbw still provides a method of doing just that.
Here are a few of the reasons why you might want to override the build
phase of a component:

\begin{itemize}
\item Test \bni Infrastructure by not building a component

  How does your \bni system function if a required component is not
  actually built?

  Will it fail during \emph{install}, or while integrating into a
  production deliverable?  Will it fail with an easy-to-diagnose
  error?

  The simle test would be to add a build override target, and then do
  nothing.

\item Test \bni Infrastructure by corrupting component build

  Will your \bni system recognize if a build has been corrupted?
  Perhaps only some files are built, or maybe the executables are
  corrupt?

  You could override the build target, build the entire component
  (using the \lmsbw infrastructure) and then corrupt the build output
  in-situ.

\item Apply a patch in-situ

  The component's source tree cannot be changed in the source tree,
  but you can apply a patch\todo{Applying patches should be part of
    \lmsbw} to the sources after they have been copied to the build
  directory.

  To do this, you'd override the build rule to apply the patches, then
  use the \lmsbw infrastructure to invoke the component's build
  process.

\item Component may not need a build; only install static files

  Perhaps the component doesn't have a build; maybe only static files
  are installed.  In this case, you could override the build to do
  nothing, and save a minute amount of time.

\item Other unknown reasons
\end{itemize}

Once you have determined that you need to override the automatic rules
to perform a component build, the actual work is quite trivial; all
you need to do is add a \makefile target to your component
configuration file, and supply the new commands to associated with the
target.  The name of the target is based on the component name; if the
component name is \texttt{alpha}, you would do the following:

\begin{verbatim}
component.build.alpha:
        echo "Overriding trampoline and not building alpha";
\end{verbatim}


\section{Overriding Install}

\begin{itemize}
\item Test \bni Infrastructure by not installing component
\item Component \makefile has no install rule, and cannot be modified
\item Component \makefile does not use \destdir
\item Component does not install all needed files
\item Other unknown reasons
\end{itemize}

\subsection{Overriding Component \makefile
  Invocation}\label{wrap:override-target}

In practice, this will be an issue with the \texttt{install} target
much more often than the \texttt{build} target, but \lmsbw provides a
facility for you to override both targets without modifying the final
\texttt{makefile}.

This override technique works because of the two-step nature of
\lmsbw.  \lmsbw proper reads configuration files, and generates \make
rules, and then standard \make rule processing takes over.  When it is
determined that a component needs to be built, \lmsbw proper invokes a
second \makefile to trampoline into the actual component \makefile.
This trampoline deals with setting up the toolchain, and handles the
building \& installtion of the component through its own set of \make
rules.  The \make rules in the tramoline \makefile are where the
actual component build process is invoked -- and it is these rules
that you can override in your component configuration file.

You can see the \texttt{sports} sample for an example of how this
works.

\subsubsection{build}\label{wrap:override-target-build}

If you add a \makefile target to your component configuration file
with the following name:
\texttt{component.build.}\texttt{<component-name>}, then it will be
used by the trampoline \makefile to build your component.

If your component configuration file does \emph{not} have this target,
the trampoline \makefile will invoke the compnent \makefile to create
the \texttt{build} target.

If you provide this target, it will be your responsibility to invoke
the component \makefile to build the component.

For example, if your component was named \texttt{alpha}, you could do
something like this:

\begin{verbatim}
component.build.alpha:
        echo "Overriding trampoline and not building alpha";

\end{verbatim}

Overriding this rule is seldom needed, and should not be done unless
there is an absolute need.

\subsubsection{install}\label{wrap:override-target-install}

If you add a \makefile target to your component configuration file
with the following name:
\texttt{component.install.}\texttt{<component-name>}, then it will be
used by the trampoline \makefile to install your component.

This can be useful if the component build process cannot be modified
and it does not install public build output.

For example, a component named \texttt{alpha} might have the install
rule overriden with something like this:

\begin{verbatim}
component.install.alpha:
        $(MKDIR) --parents $(DESTDIR)/usr/include/alpha
        $(CP) --recursive \
             $(LMSBW_C_BUILD_DIRECTORY)/include/* \
             $(DESTDIR)/usr/include/alpha
\end{verbatim}

