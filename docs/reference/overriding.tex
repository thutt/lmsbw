% Copyright (c) 2012 Taylor Hutt, Logic Magicians Software
%
% This program is free software: you can redistribute it and/or
% modify it under the terms of the GNU General Public License as
% published by the Free Software Foundation, either version 3 of the
% License, or (at your option) any later version.
%
% This program is distributed in the hope that it will be useful, but
% WITHOUT ANY WARRANTY; without even the implied warranty of
% MERCHANTABILITY or FITNESS FOR A PARTICULAR PURPOSE.  See the GNU
% General Public License for more details.
%
% You should have received a copy of the GNU General Public License
% along with this program.  If not, see <http://www.gnu.org/licenses/>.
%
\chapter{Overriding Default Rules via Configuration} \label{chap:overriding}

\subsection{Overriding Component \makefile
  Invocation}\label{wrap:override-target}

In some cases, you'll have a component \makefile that cannot be edited
for various reasons (office politics, code maintained elsewhere,
etc.), and it doesn't conform to the \lmsbw requirements.

In practice, this will be an issue with the \texttt{install} target
much more often than the \texttt{build} target, but \lmsbw provides a
facility for you to override both targets without modifying the final
\texttt{makefile}.

This override technique works because of the two-step nature of
\lmsbw.  \lmsbw proper reads configuration files, and generates \make
rules, and then standard \make rule processing takes over.  When it is
determined that a component needs to be built, \lmsbw proper invokes a
second \makefile to trampoline into the actual component \makefile.
This trampoline deals with setting up the toolchain, and handles the
building \& installtion of the component through its own set of \make
rules.  The \make rules in the tramoline \makefile are where the
actual component build process is invoked -- and it is these rules
that you can override in your component configuration file.

You can see the \texttt{sports} sample for an example of how this
works.

\subsubsection{build}\label{wrap:override-target-build}

If you add a \makefile target to your component configuration file
with the following name:
\texttt{component.build.}\texttt{<component-name>}, then it will be
used by the trampoline \makefile to build your component.

If your component configuration file does \emph{not} have this target,
the trampoline \makefile will invoke the compnent \makefile to create
the \texttt{build} target.

If you provide this target, it will be your responsibility to invoke
the component \makefile to build the component.

For example, if your component was named \texttt{alpha}, you could do
something like this:

\begin{verbatim}
component.build.alpha:
        echo "Overriding trampoline and not building alpha";

\end{verbatim}

Overriding this rule is seldom needed, and should not be done unless
there is an absolute need.

\subsubsection{install}\label{wrap:override-target-install}

If you add a \makefile target to your component configuration file
with the following name:
\texttt{component.install.}\texttt{<component-name>}, then it will be
used by the trampoline \makefile to install your component.

This can be useful if the component build process cannot be modified
and it does not install public build output.

For example, a component named \texttt{alpha} might have the install
rule overriden with something like this:

\begin{verbatim}
component.install.alpha:
        $(MKDIR) --parents $(DESTDIR)/usr/include/alpha
        $(CP) --recursive \
             $(LMSBW_C_BUILD_DIRECTORY)/include/* \
             $(DESTDIR)/usr/include/alpha
\end{verbatim}

