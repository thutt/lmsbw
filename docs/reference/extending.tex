% Copyright (c) 2012 Taylor Hutt, Logic Magicians Software
%
% This program is free software: you can redistribute it and/or
% modify it under the terms of the GNU General Public License as
% published by the Free Software Foundation, either version 3 of the
% License, or (at your option) any later version.
%
% This program is distributed in the hope that it will be useful, but
% WITHOUT ANY WARRANTY; without even the implied warranty of
% MERCHANTABILITY or FITNESS FOR A PARTICULAR PURPOSE.  See the GNU
% General Public License for more details.
%
% You should have received a copy of the GNU General Public License
% along with this program.  If not, see <http://www.gnu.org/licenses/>.
%
\chapter{Extending \lmsbw}\label{chap:extending}

\lmsbw has been designed and written to facilitate easy extension
through the adding of new component types.  For the most part, the
work required to extend will take place in the core
\texttt{wrapper/engine} code, but depending on the extension, you may
need to add some additional rules to the trampoline \texttt{makefile}
in \texttt{wrapper/component}.

Extending boils down to performing these steps:

\begin{enumerate}
\item Choose component \texttt{kind}

  The new \texttt{kind} should be a single word, all lowercase, and be
  a good mnemonic.

\item Create a report function

  The report generation (\texttt{generate-report.mk}) system generates
  reports for each component based on their \texttt{kind}.  It does
  this by calling a function named using the component's \texttt{kind}
  field.  Thus, if you have add a new component \texttt{kind} of
  \emph{download}, you'll have to create a function called:

\begin{verbatim}
generate_component_report_download
\end{verbatim}

  The function should take one argument, and the actual argument
  passed to it will be a \texttt{LMSBW\_component\_<component>}
  (\xref{variables:lmsbw-component-component}).

  Examining \texttt{generate\_component\_report\_source} will show the
  way that this function should be implemented.

\item Create a rule generator function

  Dependency rule generation in \lmsbw is handled by calling a
  function based on the \texttt{kind} of each component.  If you add a
  new component \texttt{kind} of \emph{download}, you'll have to
  create a function called:

\begin{verbatim}
generate_component_rules_download
\end{verbatim}

  The function should take one argument, and the actual argument
  passed to it will be a \texttt{LMSBW\_component\_<component>}
  (\xref{variables:lmsbw-component-component}).

  Examining \texttt{generate\_component\_rules\_source} will show the
  way that this function should be implemented.

\item Update \texttt{component.makefile}

  If your new component \texttt{kind} requires the component's build
  process to be invoked for other reasons -- for example, maybe to run
  \texttt{./configure} -- you will have to update
  \texttt{component.makefile} to support this new trampoline mode.

\item Update \lmsbwconfiguration

  If your new component \texttt{kind} requires global configuration
  changes, make them using the \lmsbwconfiguration
  (\xref{variables:lmsbw-configuration}) variable.

\item Update documentation

  At the very least, you'll need to update \xref{chap:variables} and
  \xref{chap:api} to reflect your additions.

\item Write new tests

  You should write new tests to ensure that your changes are correct.
  Update \texttt{lmsbw-run-tests} to incorporate them.

\item Run tests

  The \texttt{lmsbw-run-tests} script should be run.

\item Update \texttt{generate} sample.

  You must update the \texttt{generate} sample to incorporate your
  new component \texttt{kind}.

  Your new component \texttt{kind} should not be significantly
  different in original build time than the \makefile version.

  Your new component \texttt{kind} must not alter the incremental
  characteristics of \lmsbw.

\item Add new samples

  You must add new samples that show how to declare and use the new
  component \texttt{kind}.

\end{enumerate}

