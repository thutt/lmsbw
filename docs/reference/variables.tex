% Copyright (c) 2012 Taylor Hutt, Logic Magicians Software
%
% This program is free software: you can redistribute it and/or
% modify it under the terms of the GNU General Public License as
% published by the Free Software Foundation, either version 3 of the
% License, or (at your option) any later version.
%
% This program is distributed in the hope that it will be useful, but
% WITHOUT ANY WARRANTY; without even the implied warranty of
% MERCHANTABILITY or FITNESS FOR A PARTICULAR PURPOSE.  See the GNU
% General Public License for more details.
%
% You should have received a copy of the GNU General Public License
% along with this program.  If not, see <http://www.gnu.org/licenses/>.
%
\chapter{Variable Reference}\label{chap:variables}

The \lmsbw system uses \gmsl to simplify the implementation of data
structures in \make by using associative arrays as a kind of record
structure.  This facilitates a programming interface that uses few
variables in a uniform way that makes using \lmsbw simple.

This chapter provides a reference for the variables that are used
internally by \lmsbw, and that provide the low-level interface used to
declare and define components.

\section{\lmsbwconfiguration}\label{variables:lmsbw-configuration}

This variable is the nexus of all global configuration for a wrapped
build.  It must be initialized by your master configuration file
(\xref{lmsbw:configuration}, \xref{wrap:master-configuration}), or the
build will fail with an error message indicating that the
configuration is bad.

It is implemented as a \gmsl associative array, and is readily
extensible by simply adding new \emph{key} / \emph{value} pairs.

Its keys and values must not be changed after the initial load of the
configuration file.

Each key in this array covers a global configuration option that will
affect the entire wrapped build.  The following sections describe the
valid keys and their allowed values.

\begin{quote}
Be aware that spaces in \gnumake function arguments are significant.
The examples shown in this chapter have been formatted to be easily
readable in a printed document; it is frequently best of have the
entire function call on a single line in the source file, or to
fastidiously use \texttt{\$(strip)} on function arguments.
\end{quote}

\subsection{\texttt{component-build-support}}\label{variables:component-build-support}

This key holds a list of supported components build methods that you
want to be able to access in your wrapped build.  Loading the build
methods on-the-fly allows \lmsbw to be extensible, and ensures that
\lmsbw will only load \make functions to support the component types
you will be using.

The following values are valid:

\begin{itemize}
\item source

  The source code is available on the local system.  \lmsbw will
  manage the invocation of the \make-based build process contained
  in the source directory.

\end{itemize}

If no component kinds are specified in the configuration, \lmsbw will
fail with an error.

Only the specified configuration kinds will be loaded and accessible
when using \lmsbw.  This is the mechanism whereby support for
additional types of components can be added to \lmsbw
(\xref{chap:extending}).

\subsection{\texttt{load-configuration-function}}\label{variables:load-configuration-function}

This key must be the name of a \gnumake function which will load the
component configurations which comprise your project.

\begin{verbatim}
vv:=$(call set,LMSBW_configuration, \
      load-configuration-function,  \
      load_configuration)
\end{verbatim}

In the example above, the user-supplied function called
\texttt{load\_configuration} will be invoked to load the component
configurations.

Upon successful exit, the named function must have done the following:

\begin{itemize}
\item Create list of loaded components
  (\xref{variables:lmsbw-components})

\item Configure individual components

  For each component contained in the \lmsbwcomponents variable, there
  must be a corresponding associative array named
  \lmsbwcomponent{<component>}.  See \xref{chap:wrapping} for details
  on declaring components.

\end{itemize}


\section{\texttt{LMSBW\_components}}\label{variables:lmsbw-components}

This variable is a normal \make variable.  After the
\texttt{load-configuration-function}
(\xref{variables:load-configuration-function}) has completed
initializing the build configuration, this variable must contain list
of configured components.

The variable is internally managed when you use the official APIs for
declaring components -- you should never have to change this variable.
If you write new APIs to load or declare components, you must ensure
that this variable is updated with each load.

The list must adhere to the following rules:

\begin{itemize}
\item Space separated.
\item No duplicates.
\item Each item must have a corresponding \lmsbwcomponent{<component>}
  associative array.
\end{itemize}

\lmsbw will use this list of components to generate the rules which
will be used to build the component.

% To verify that this documentation is complete, you must find all
% component associative array keys.  This is done by looking for
% the function which sets component keys -- lmsbw_scf -- in the
% 'wrapper' directory.
\section{\lmsbwcomponent{<component>}}\label{variables:lmsbw-component-component}

Each component named in \lmsbwcomponents must have a corresponding
associative array which describes all the attributes needed to
correctly build it.  However, the allowed attributes will vary based
on the \emph{kind} (\xref{variables:kind}) of the component.

The \make variable for each \emph{kind} of source component will be
named as shown below; consider that the component name is \texttt{baselib}.

\begin{verbatim}
LMSBW_component_baselib
\end{verbatim}

Once key:value pairs are set they should not be changed.

The following sections describe the various kinds of components, their
allowable associative keys and values.

\section{Source Components}

Source components are declared to \lmsbw through
\texttt{declare\_source\_component}
(\xref{api:declare-source-component}) and have the keys which are
described in the following sections.

\subsection{\lmsbwcomponentfield{<component>}{api}}\label{variables:api}

This is a list of install directories that contain the public API of
the component.  The directories must be installed into the \destdir
during the installation phase of the build.

All directly dependent components will be automatically rebuilt if any
files change in the named directories.

This variable should be set to the directories that contain exported
header files, static libraries, icons, and other build output that
affects the building of other components.

Directories which contain shared libraries do not need to be included;
changing a shared library does not need to cause other, directly
dependent, components to be rebuilt.

If there is no exported API -- no header files, no static libraries,
etc. -- then you do not need to set this key.

\apiref{api:component-attribute-api}

\subsection{\lmsbwcomponentfield{<component>}{build-directory}}\label{variables:build-directory}

This key is used internally; it contains the absolute pathname of the
directory in which the build will be performed.

This value should not be used by a client of \lmsbw.

\apiref{api:declare-source-component}

\subsection{\lmsbwcomponentfield{<component>}{build-log}}\label{variables:build-log}

This key is used internally; it contains the absolute pathname of the
console output of the component's build process.

If you want to view the log for a component, use the \texttt{log}
verb, like so:

\begin{verbatim}
lmbw log.hello-world
\end{verbatim}

This value should not be used by a client of \lmsbw.

\apiref{api:declare-source-component}

\subsection{\lmsbwcomponentfield{<component>}{build-root-directory}}\label{variables:build-root-directory}

This key holds the root directory where all build-related output is
placed for the associated component.

This value should not be used by a client of \lmsbw.

\apiref{api:declare-source-component}

\subsection{\lmsbwcomponentfield{<component>}{build-target}}\label{variables:build-target}

This key holds the target used when invoking the component's \makefile
to \emph{build}.

If not specified, the component's build system will invoked with no
target when building; this will cause the default target of the
component's \makefile to be built.

This is included because it is sometimes necessary to not fulfill the
default target of the component's build process -- if you are cross
compiling, for example, and the component build process attempts to
execute tests that were just built for another architecture.

Compare to \xref{variables:install-target}.

\apiref{api:component-attribute-build-target}\label{variables:build-target}

\subsection{\lmsbwcomponentfield{<component>}{cflags}}\label{variables:cflags}

This key holds values which will be assigned to the \texttt{CFLAGS}
variable before the component's \makefile is invoked.

The value of this key is also used to determine the build directory
for each component.  Changing the value for a component will change
the build output directory; the purpose of this several fold:

\begin{itemize}
\item \texttt{CFLAGS} changes incorporated into build

  Build processes infrequently have correct dependencies to do a
  full-rebuild when compiler options change.  This method of assigning
  a build directory (partially) based on the compiler options
  guarantees that option changes are correctly reflected in the build.

\item Incremental builds

  Changing the options will result in the build output directory
  changing, but changing them back to their original value will also
  restore the old build directory; incremental builds between compiler
  options changes remain fast.

\item Shared build output

  A component compiled with the same global options, regardless if it
  is for a different SKU, will share the same build output.

\end{itemize}

It is recommended that all compiler options are set in the component
configuration file rather than in the component \makefile.

\apiref{api:cflags}

\subsection{\lmsbwcomponentfield{<component>}{component}}\label{variables:component}

This key contains the name of the component.

\apiref{api:declare-source-component}

\subsection{\lmsbwcomponentfield{<component>}{configuration-file}}\label{variables:configuration-file}

This value is the absolute pathname of the configuration file which
was used to declare the component.

Changes to the configuration file will result in the component's build
process being invoked; whether or not this results in a actual files
being rebuilt depends on the nature of the changes to the
configuration file, and the component's \makefile.  Changing a
comment, for example, will not result in any files being rebuilt, but
changing \texttt{CFLAGS} (\xref{variables:cflags}) will.

Furthermore, configuration files can contain \makefile rules which
will override the default rules for building and installing the
component (\xref{chap:overriding}).

% Move this into the overriding chapter...
%
% For example, if the wrapped build system does not have an
% \texttt{install} rule, this can be provided by an overriding rule in
% the configuration file.  This overriding rule would copy files from
% known locations out of the build directory and install them into the
% proper locations in the \destdir directory.  See
% \xref{chap:overriding}.

\apiref{api:declare-source-component}

\subsection{\lmsbwcomponentfield{<component>}{description}}\label{variables:description}

This key holds a brief description of the component.

\apiref{api:declare-source-component}

\subsection{\lmsbwcomponentfield{<component>}{destdir-directory}}\label{variables:destdir-directory}

This key holds the absolute path of the \destdir directory; after
building, the component's public files must be \emph{installed} to this
intermediate install directory via the \texttt{install} target of the
component's \makefile.

\lmsbw proper will then copy the files from the intermediate directory
to the \texttt{productroot}\todo{glossary 'productroot'} directory.

To install files, append the acutal install pathname to \destdir.  For
example, if your file should end up in \texttt{/usr/include}, one way
to install it would be:

\begin{verbatim}
cp include-file.h $(DESTDIR)/usr/include
\end{verbatim}

\apiref{api:declare-source-component}

\subsection{\lmsbwcomponentfield{<component>}{install-directory}}\label{variables:install-directory}

This key holds the absolute path of the component's view of the
\emph{install} directory.  After building, \lmsbw will invoke the
component's build process with the \emph{install} target, and this
target must place files into the \destdir directory.  Upon successful
completion of component installation, \lmsbw will copy the files from
the \destdir to the global \emph{install} directory; the global
install directory, \texttt{productroot}, is shared by all components.

This value of this key must not be used by components of \lmsbw.

\apiref{api:declare-source-component}

\subsection{\lmsbwcomponentfield{<component>}{install-target}}\label{variables:install-target}

This key holds the component's \makefile target that is used to
\emph{install} the component into the \destdir.  Compare to
\xref{variables:build-target}.

If not specified, the component's build system will invoked with
\texttt{install} as the target when processing the \emph{install}
phase of the build.

\apiref{api:component-attribute-install-target}\label{variables:install-target}

\subsection{\lmsbwcomponentfield{<component>}{kind}}\label{variables:kind}

This key represents the \emph{kind} of the component.

See \xref{api:kind} for the allowable values and semantics.

\apiref{api:declare-source-component}

\subsection{\lmsbwcomponentfield{<component>}{no-parallel}}\label{variables:no-parallel}

This key, when set, causes the recursive make for the associated
component to \emph{never} allow parallel builds; it directly sets the
\gnumake parallel option to \texttt{-j 1}.

This key is normally only needed to address build issues -- caused by
bad dependencies -- in a very poorly constructed \makefile.

\apiref{api:component-attribute-no-parallel-build}

\subsection{\lmsbwcomponentfield{<component>}{prerequisite}}\label{variables:prerequisite}

This key holds a list of components declared to be prerequisites of
the associated component.  All components contained in this list will
be successfully built and installed before the associated component is
even attempted to be built.

An empty value means that there are no prerequisites.

\apiref{api:declare-source-component}

\subsection{\lmsbwcomponentfield{<component>}{reason}}\label{variables:reason}

This key holds the \emph{reason} this component is being built.

See \xref{api:reason} for the allowable values and semantics.

\apiref{api:declare-source-component}

\subsection{\lmsbwcomponentfield{<component>}{source-directory}}\label{variables:source-directory}

This key contains the absolute pathname of the source directory for
the component.

This value should not be used by a client of \lmsbw.

\apiref{api:declare-source-component}

\subsection{\lmsbwcomponentfield{<component>}{source-mtree-manifest}}\label{variables:source-mtree-manifest}

This key holds the absolute pathname of the file which holds the
\mtree manifest for the source directory.

This value should not be used by a client of \lmsbw.

\apiref{api:declare-source-component}

\subsection{\lmsbwcomponentfield{<component>}{toolchain}}\label{variables:toolchain}

This is the name of the toolchain that will be used to build the
component.  It can be specified at the component level.  If it is not
specified at the component level, the globally specified toolchain is
used.  If no global toolchain is specified, the host toolchain will be
used.

\apiref{api:toolchain}
