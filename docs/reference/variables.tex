% Copyright (c) 2012 Taylor Hutt, Logic Magicians Software
%
% This program is free software: you can redistribute it and/or
% modify it under the terms of the GNU General Public License as
% published by the Free Software Foundation, either version 3 of the
% License, or (at your option) any later version.
%
% This program is distributed in the hope that it will be useful, but
% WITHOUT ANY WARRANTY; without even the implied warranty of
% MERCHANTABILITY or FITNESS FOR A PARTICULAR PURPOSE.  See the GNU
% General Public License for more details.
%
% You should have received a copy of the GNU General Public License
% along with this program.  If not, see <http://www.gnu.org/licenses/>.
%
\chapter{Variable Reference}

The \lmsbw system uses \gmsl to simplify the implementation of data
structures in \make by using associative arrays as a kind of record
structure.  This facilitates a programming interface that uses few
variables in a uniform way that makes using \lmsbw simple.

This chapter provides a reference for the variables that are used
internally by \lmsbw, and that provide the low-level interface used to
declare and define components\todo{glossary:component, module}.

\section{\lmsbwconfiguration}

This variable is the nexus of all global configuration for a wrapped
build.  It must be initialized by the file referenced via the \lmsbw
\texttt{--configuration} option (\xref{lmsbw:configuration}), or the
build will fail with an error message indicating that the
configuration is bad.

It is implemented via an associative array provided by \gmsl, as as
such it is readily extensible by simply adding new \emph{key} /
\emph{value} pairs.

Each key in this array covers a global configuration option that will
affect the entire wrapped build, and its keys and values must only be
modified during the initial load of the configuration file.

The following sections describe the valid keys and their allowed
values.

Be aware that spaces in \gnumake function arguments are significant.
The examples shown in this chapter have been formatted to be easily
readable in a printed document; it is frequently best of have the
entire function call on a single line in the source file.\todo{This
  should be in a 'display' box}


\subsection{\texttt{load-configuration-function}}\label{aa:load-configuration-function}

The value of this key must be the name of a \gnumake function which
will initialize the configuration of components which comprise the
wrapped build.

\begin{verbatim}
vv:=$(call set,LMSBW_configuration, \
      load-configuration-function,  \
      load_configuration)
\end{verbatim}

In the example above, the function called \texttt{load\_configuration}
will be invoked to create the configuration of components.

At exit, the named function must have done the following:

\begin{itemize}
\item Create list of loaded components

  \lmsbwcomponents must be a space-separated, no duplicate, list of
  configured components.
\item Configure individual components

  For each component contained in the \lmsbwcomponents variable, there
  must be a corresonding associative array named
  \lmsbwcomponent{<component>}.

\end{itemize}

\section{\texttt{LMSBW\_components}}

This variable is a normal \make variable.  After the
\texttt{load-configuration-function}
(\xref{aa:load-configuration-function}) has completed initializing the
build configuration, this variable must contain a list of configured
components.

The list must adhere to the following rules:

\begin{itemize}
\item Space separated.
\item No duplicates.
\item Each item must have a corresponding \lmsbwcomponent{<component>}
  associative array.
\end{itemize}

\lmsbw will use this list of components to generate the rules which
will be used to build the component.

\section{\lmsbwcomponent{<component>}}

Each component named in \lmsbwcomponents must have a corresponding
associative array which describes all the attrbutes needed to
correctly build it.

The following sections describe the allowable keys, and their allowed
values.

\subsection{\lmsbwcomponentfield{<component>}{binary-api}}\label{lmsbwcomponent:binary-api}

This key contains the full pathname of the \emph{install} directory
that contains the binary interface of the component.  \lmsbw will use
the \destdir version of the install directory to perform \mtree
manifest creation.

The intended use of this key is:

\begin{itemize}
\item statically linked libraries
\item binary blobs which are included in other programs
\end{itemize}

Unlike the source API (\xref{lmsbwcomponent:souce-api}), which should
change very infrequently, the binary API is expected to change with
each compilation of the component which produces it; requiring a full
rebuild of dependent components would be objectionable.

Instead of requiring a full recopmile, if the binary API of a
prerequisite component has changed, \lmsbw will recurse and build the
dependent component: it is the responsibility of the dependent
component to have the proper dependencies to correctly utilize the new
binary API.

This key is normally set through the provided wrapping API
(\xref{chap:wrapping}).

\subsection{\lmsbwcomponentfield{<component>}{binary-api-changed}}

This key is used internally when it has been determined that the the
binary API (\xref{lmsbwcomponent:binary-api}) has changed.

The value is the name of the file that indicates the binary API has
changed.

When this particular file exists, the \lmsbw will always rebuild
dependent components.

This value should not be used by a client of \lmsbw.

\subsection{\lmsbwcomponentfield{<component>}{binary-api-mtree-manifest}}

This key holds the absolute pathname of the file which holds the
binary API \mtree manifest.

This file is created against the files in the source API directory
(\xref{lmsbwcomponent:binary-api}).  If there is no declared binary API
directory, there will be no manifest.

This value should not be used by a client of \lmsbw.

\subsection{\lmsbwcomponentfield{<component>}{build-directory}}

This key is used internally; it contains the absolute pathname of the
directory in which the build will be performed.

This value should not be used by a client of \lmsbw.

\subsection{\lmsbwcomponentfield{<component>}{build-log}}

This key is used internally; it contains the absolute pathname of the
output of the component's bulid process.

If you want to view the log for a component, use the \texttt{log}
verb, like so:

\begin{verbatim}
lmbw log.hello-world
\end{verbatim}

This value should not be used by a client of \lmsbw.

\subsection{\lmsbwcomponentfield{<component>}{build-root-directory}}

This key holds the root directory where all build-related output is
placed for the associated component.

This value should not be used by a client of \lmsbw.

\subsection{\lmsbwcomponentfield{<component>}{build-target}}

This key holds the value specified with the
\texttt{declare\_component\_build\_target} API.

It is used to specify the target to execute inside the component's
build system of to build, but not install, the component.

If not specified, the component's build system will invoked with no
target when building; this will cause the default target of the build
system to be executed.

This is important because sometimes it is necessary to not execute the
entire default target -- if you are cross compiling, for example, and
the component executes tests using the program that was just build.

\subsection{\lmsbwcomponentfield{<component>}{component}}

This key holds the value specified with the
\texttt{declare\_component\_component} API.

This field contains the name of the component.

\subsection{\lmsbwcomponentfield{<component>}{configuration-file}}

This key holds the value specified with the
\texttt{declare\_component\_configuration\_file} API.

This value is the name of the configuration file which was used to
declare the component.  Configuration files can contain \makefile
rules which will override the default rules for building and
installing the component.

For example, if the wrapped build system does not have an
\texttt{install} rule, this can be provided by an overriding rule in
the configuration file.  This overriding rule would copy files from
known locations out of the build directory and install them into the
proper locations in the \destdir directory.  See
\xref{chap:overriding}.

\subsection{\lmsbwcomponentfield{<component>}{description}}
\subsection{\lmsbwcomponentfield{<component>}{destdir-directory}}

This value should not be used by a client of \lmsbw.

\subsection{\lmsbwcomponentfield{<component>}{direct-dependents}}

This value should not be used by a client of \lmsbw.

\subsection{\lmsbwcomponentfield{<component>}{install-directory}}

This value should not be used by a client of \lmsbw.

\subsection{\lmsbwcomponentfield{<component>}{install-target}}
\subsection{\lmsbwcomponentfield{<component>}{kind}}
\subsection{\lmsbwcomponentfield{<component>}{module}}
\subsection{\lmsbwcomponentfield{<component>}{no-parallel}}
\subsection{\lmsbwcomponentfield{<component>}{prerequisite}}
\subsection{\lmsbwcomponentfield{<component>}{reason}}
\subsection{\lmsbwcomponentfield{<component>}{source-api}}\label{lmsbwcomponent:souce-api}

\subsection{\lmsbwcomponentfield{<component>}{source-api-changed}}

This key is used internally when it has been determined that the the
source API (\xref{lmsbwcomponent:souce-api}) has changed.

The value is the name of the file that indicates the source API has
changed.

When this particular file exists, the \lmsbw will fail with an error
indicating that your change affect dependent components, and it shows
a command to execute which will cause all the dependen components to
have the verb \texttt{clean} applied to their build.  This ensures
that a full rebuild of all dependent components will be performed,
thereby avoiding any dependency issues which would have caused the
dependent components to not automatically properly rebuild.

This also gives the developer a chance to pause and think about the
total impact -- other required source-level changes, additional
testing, etc. -- of their change.

This value should not be used by a client of \lmsbw.

\subsection{\lmsbwcomponentfield{<component>}{source-api-mtree-manifest}}

This key holds the absolute pathname of the file which holds the
source API \mtree manifest.

This file is created against the files in the source API directory
(\xref{lmsbwcomponent:souce-api}).  If there is no declared source API
directory, there will be no manifest.

This value should not be used by a client of \lmsbw.

\subsection{\lmsbwcomponentfield{<component>}{source-directory}}
\subsection{\lmsbwcomponentfield{<component>}{source-mtree-manifest}}

This value should not be used by a client of \lmsbw.


