% Copyright (c) 2012 Taylor Hutt, Logic Magicians Software
%
% This program is free software: you can redistribute it and/or
% modify it under the terms of the GNU General Public License as
% published by the Free Software Foundation, either version 3 of the
% License, or (at your option) any later version.
%
% This program is distributed in the hope that it will be useful, but
% WITHOUT ANY WARRANTY; without even the implied warranty of
% MERCHANTABILITY or FITNESS FOR A PARTICULAR PURPOSE.  See the GNU
% General Public License for more details.
%
% You should have received a copy of the GNU General Public License
% along with this program.  If not, see <http://www.gnu.org/licenses/>.
%
\chapter{Tips \& Tricks}

This chapter describes tips \& tricks that are not obvious from
reading the reference manual.

\section{Global Toolchain \& \texttt{build} Components}

The global toolchain (\xref{usinglmsbw:toolchain}) is never used for
\texttt{build} components.  If the toolchain is not specified at the
component level, then the host toolchain will be used.

\section{Install Directory Differentiation}

The following are two features that contribute to the incremental
speed of \lmsbw.

\begin{itemize}
\item Shared Build Output

  When possible, \lmsbw will share the build output of a component
  between different master configuration files.  This is useful to
  reduce build time if the different master configuration files
  produce different SKUs of the same product.

\item Install Directory Location

  The name of the install directory is computed based on the
  following:

  \begin{itemize}
  \item{The pathname of the master configuration file}
  \item{The set of loaded components}
  \item{The global toolchain}
  \end{itemize}
\end{itemize}

In many cases, this mechanism works flawlessly, but there is a
particular method of configuration that is not recommended.

\begin{itemize}
\item{Different SKUs are produced}
\item{Same master configuration file for all SKUs}
\item{Same set of components for each product}
\item{Component build output differs via conditional compilation}
\end{itemize}

In the case described above, the configuration of each component is
usually controlled via environment variables that are proxied to the
toolchain and used in conditional compilation.  However, due to the
way \lmsbw is written to choose the install directory, this mode of
configuration will cause each SKU to share the same install directory.

To ensure that each SKU has its own install directory, at least one of
the criteria used to create the install directory must be changed.
The following recommendations are easy to follow:

\begin{itemize}
\item Different Master Configurations

  Create physically different master configuration files; symlinks
  will not suffice because \lmsbw resolves symlinks to the absolute
  pathname of the file.

  One trick is to make a common configuration file that is included by
  the differently-named master configuration files.

\item{Different Component Sets}

  Ensure that each SKU has a different set of components.
\end{itemize}

\section{Master Configurations for More Speed}

If the startup time of \lmsbw is longer than the incremental build of
any part of the project, it is possible to tune the configuration to
regain the lost incremental development speed.

\begin{itemize}
\item New Master Configuration

  Create a new master configuration which contains only the desired
  component and necessary perquisites.

  Because \lmsbw shares build output, the \emph{small} master
  configuration will cause the component to be built in the same
  directory as the \emph{original} master configuration.

\item Incrementally Build

  Once the new master configuration is set up, incrementally build the
  component with a much smaller startup time.

  If the component can be tested stand-alone, it is also feasible to
  test using this smaller master configuration.

\item Switch Back to \emph{Original} Master Configuration

  Once work has been completed and verified, switch back to the
  \emph{original} master configuration file, and re-build.

  Since the sub-project is already built in the shared build
  directory, the only work \lmsbw needs to perform will be to install
  the component into the \emph{original} configuration's install
  directory.
\end{itemize}

