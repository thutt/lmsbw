% Copyright (c) 2012 Taylor Hutt, Logic Magicians Software
%
% This program is free software: you can redistribute it and/or
% modify it under the terms of the GNU General Public License as
% published by the Free Software Foundation, either version 3 of the
% License, or (at your option) any later version.
%
% This program is distributed in the hope that it will be useful, but
% WITHOUT ANY WARRANTY; without even the implied warranty of
% MERCHANTABILITY or FITNESS FOR A PARTICULAR PURPOSE.  See the GNU
% General Public License for more details.
%
% You should have received a copy of the GNU General Public License
% along with this program.  If not, see <http://www.gnu.org/licenses/>.
%
\chapter{Tips \& Tricks}

This chapter describes tips \& tricks that are not obvious from
reading the reference manual.

\section{Global Toolchain \& \texttt{build} Components}

The global toolchain (\xref{usinglmsbw:toolchain}) is never used for
\texttt{build} components.  If the toolchain is not specified at the
component level, then the host toolchain will be used.

\section{Install Directory Differentiation}\label{tips:install-directory}

The following are two features that contribute to the incremental
speed of \lmsbw.

\begin{itemize}
\item Shared Build Output

  When possible, \lmsbw will share the build output of a component
  between different master configuration files.  This is useful to
  reduce build time if the different master configuration files
  produce different SKUs of the same product.

\item Install Directory Location (\xref{variables:install-directory})

  \lmsbw handles the install directory location, automatically
  changing it when the geometry -- configuration file, components
  included, toolchain used -- of the project changes.

\end{itemize}

In many cases, this mechanism works flawlessly, but there is a
particular method of configuration that is not recommended.

\begin{itemize}
\item{Different SKUs are produced}
\item{Same master configuration file for all SKUs}
\item{Same set of components for each product}
\item{Component build output differs via conditional compilation}
\end{itemize}

In the case described above, the configuration of each component is
usually controlled via environment variables that are proxied to the
toolchain and used in conditional compilation.  However, due to the
way \lmsbw is written to choose the install directory, this mode of
configuration will cause each SKU to share the same install directory.

To ensure that each SKU has its own install directory, at least one of
the criteria used to create the install directory must be changed.
The following recommendations are easy to follow:

\begin{itemize}
\item Different Master Configurations

  Create physically different master configuration files; symlinks
  will not suffice because \lmsbw resolves symlinks to the absolute
  pathname of the file.

  One trick is to make a common configuration file that is included by
  the differently-named master configuration files.

\item{Different Component Sets}

  Ensure that each SKU has a different set of components.
\end{itemize}

\section{Master Configurations for More Speed}

If the startup time of \lmsbw is longer than the incremental build of
any part of the project, it is possible to tune the configuration to
regain the lost incremental development speed.

\begin{itemize}
\item New Master Configuration

  Create a new master configuration which contains only the desired
  component and necessary perquisites.

  Because \lmsbw shares build output, the \emph{small} master
  configuration will cause the component to be built in the same
  directory as the \emph{original} master configuration.

\item Incrementally Build

  Once the new master configuration is set up, incrementally build the
  component with a much smaller startup time.

  If the component can be tested stand-alone, it is also feasible to
  test using this smaller master configuration.

\item Switch Back to \emph{Original} Master Configuration

  Once work has been completed and verified, switch back to the
  \emph{original} master configuration file, and re-build.

  Since the sub-project is already built in the shared build
  directory, the only work \lmsbw needs to perform will be to install
  the component into the \emph{original} configuration's install
  directory.
\end{itemize}

\section{Finding Build Output}

Although \lmsbw can place build \& install output in different
directories, seemingly arbitrarily, that should not be a major concern
because \lmsbw will also produce the locations.

\subsection{\texttt{installdir}}

The \lmsbw \texttt{installdir} verb allows the inspection of the
current install directory for every component, or for a single
component.

See \xref{usinglmsbw:installdir} for information on using the verb.

See \xref{variables:install-directory} for information on the
selection criteria for the directory.

\subsection{\texttt{builddir}}

The \lmsbw \texttt{builddir} verb allows the inspection of
the current build directory for every component, or for a single
component.

See \xref{usinglmsbw:builddir} for information on using the verb.

See \xref{variables:build-directory} for the criteria that are used to
determine the directory name.

\subsection{\destdir}
The \lmsbw \texttt{destdir} verb allows the inspection of
the current build directory for every component, or for a single
component.

See \xref{usinglmsbw:destdir} for information on using the verb.

See \xref{variables:destdir-directory} for the criteria that are used
to determine the directory name.

\section{Uniquely Identifying Source Code}
\label{tips:uniquely-identify-source-code}

To be able to download the build output of components, it must first
be possible to match the current state of the component's source code
with an officially produced build.  In other words, the state of the
compnent's source code must be uniquely identifiable -- each change to
the source code must produce a different unique identifier.

The following are examples of unique identifiers:

\begin{itemize}
\item{Perforce change number}
\item{git sha1}
\item{md5sum of sources}
\item{sha1sum of sources}
\end{itemize}

\lmsbw defines that the
\texttt{component-build-output-download-script}
(\xref{variables:component-build-output-download-script}) does the
following:

  \begin{description}
  \item[recommended] Determine if the source tree has been
    locally modified.

  \item[required] Determine if the source tree exactly matches an
    official build.
  \end{description}

When the source tree does not exactly match an official build, the
build output cannot be downloaded.  More specifically, if the source
has been locally modified, and that fact can be identified, the
developer will be notified that they have modified a component that
will not be built, and they can then take corrective action (See
\xref{usinglmsbw:disable-build-output-download} and
\xref{usinglmsbw:no-download}).

The method employed to uniquely identify the state of the source tree
is ultimately determined by the build team, but \lmsbw provides a
script which will produce a single SHA hash for all the files in the
source directory tree.  This is called:

\begin{verbatim}
${LMSBW_SCRIPT_DIRECTORY}/lmsbw-hash-component
\end{verbatim}

This script is used by the \texttt{build-output-download} sample to
identify official builds.

\section{Downloading Build Output}
\label{tips:build-output-download-script}

\lmsbw does not directly provide a mechanism to download build output
into the build tree; that support must be provided in the way of a
script by the build team.  The \texttt{build-output-download} sample
provides an example script that shows the semantics that must be
followed.

Once the component build and archive have been created (see
\xref{tips:using-build-output-download}), the following steps must be
undertaken to download the build output:

\begin{enumerate}
\item{Uniquely Identify Sources}

  See \xref{tips:uniquely-identify-source-code}.

\item Obtain build output

  If the unique source identifier matches an official build, download
  the archive of the build output.

\item{Determine \destdir}

  The component's \destdir directory is where the build output must be
  placed.

  use the \texttt{destdir.<component-name>} \lmsbwcmd target to find
  the component's destdir.

\item{Extract Archive}

  The archive should be extracted into the determined \destdir.

  The act of extracting the archive should create the directory
  structure in the \destdir exactly as if the component build had been
  invoked.

\item{Delete Archive}

  Be sure to clean up by deleting the downloaded archive.

\end{enumerate}

\section{Using \texttt{build-output-download}}
\label{tips:using-build-output-download}

Downloading build output can be used to speed up developer's local
builds, but the amount of work necessary to enable this feature is not
clearly spelled out in the reference manual.  This section provides a
punch list that enumerates the work necessary to use the feature.

Note that the \texttt{build-output-download} sample provides an
example of all the required machinery.

\begin{enumerate}
\item{Uniquely Identify Source Tree Contents}

  See \xref{tips:uniquely-identify-source-code} for details on how the
  component's source must be identified.

\item{Download Script}

  A script to download build output must be provided.  See
  \xref{tips:build-output-download-script} for details.

\item{Component Configuration}

 Each componet that is to have its build output downloaded must be
 attributed using
 \texttt{component\_attribute\_build\_output\_download}
 (\xref{api:build-output-download}).

 Once the component is attributed, each subsequent build will attempt
 to download the component's build output, unless either of the
 following is used:

 \begin{itemize}
 \item \texttt{--no-download} (\xref{usinglmsbw:no-download})
 \item \texttt{--disable-build-output-download}
   (\xref{usinglmsbw:disable-build-output-download}).
 \end{itemize}

\item{Build Components}

  To build components which have been attributed to have their buid
  output downloaded, the following argument must be provided to
  \lmsbwcmd:

 \begin{itemize}
 \item \texttt{--disable-build-output-download}
   (\xref{usinglmsbw:disable-build-output-download}).
 \end{itemize}

 It is intended that this option will be applied when building the
 project using an official build server.

\item{Archive Component Output}

  Once the project is built, each component attributed to have the
  build output downloaded must have its \destdir archived.  This can
  be programmatically accomplished by:

  \begin{enumerate}
    \item{Determine Components}

      The \texttt{build-output-download-components} \lmsbwcmd target
      must be used to identify all components which have been
      attributed.

      \texttt{--disable-build-output-download} must \emph{not} be
      applied when this target is invoked.

    \item{Determine \destdir}

      For each attributed component, use the
      \texttt{destdir.<component-name>} \lmsbwcmd target.

      \texttt{--disable-build-output-download} \emph{must} be applied
      when this target is invoked.

    \item Determine Source Directory

      For each attributed component, use the
      \texttt{sourcedir.<component-name>} \lmsbwcmd target.

      \texttt{--disable-build-output-download} can, but need not, be
      applied when this target is invoked.

    \item{Uniquely Identify Source}

      For each attributed component, the source directory must be
      uniquely identified.  See
      \xref{tips:uniquely-identify-source-code} for details on how the
      component's source must be identified.

    \item{Archive}

      Using the determined \destdir and unique identifier, archive the
      contents of the \destdir.

      The archive must be created so that it can be extracted into the
      current working directory and the extraction should duplicate
      the directory tree beneath the original \destdir; in other
      words, the archive should be created from within the component's
      \destdir directory.

  \end{enumerate}

\item{Changing a downloaded component}

  Once a component is attributed to have its build output downloaded,
  under normal build circumstances it will never be built from source,
  and the script which downloads build output must produce an error if
  local modifications are made to the source.

  To make changes to an attributed component's source, and have the
  component built from source, use the \texttt{--no-download}
  (\xref{usinglmsbw:no-download}) argument of \lmsbwcmd.

  If a component named with \texttt{--no-download} is a prerequisite
  to other components marked to have their build output downloaded,
  those components, too, must be named with \texttt{--no-download}.

  The following \lmsbwcmd targets can be helpful in determining the
  full set of components that are dependent upon the original download
  component.

  \begin{itemize}
  \item{prerequisite-report} (\xref{usinglmsbw:prerequisite-report})
  \item{dependency-report} (\xref{usinglmsbw:dependent-report})
  \end{itemize}
\end{enumerate}

Use dependency targets to determine the set of components which are
dependent upon the one which needs to be built.

