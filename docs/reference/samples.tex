% Copyright (c) 2012 Taylor Hutt, Logic Magicians Software
%
% This program is free software: you can redistribute it and/or
% modify it under the terms of the GNU General Public License as
% published by the Free Software Foundation, either version 3 of the
% License, or (at your option) any later version.
%
% This program is distributed in the hope that it will be useful, but
% WITHOUT ANY WARRANTY; without even the implied warranty of
% MERCHANTABILITY or FITNESS FOR A PARTICULAR PURPOSE.  See the GNU
% General Public License for more details.
%
% You should have received a copy of the GNU General Public License
% along with this program.  If not, see <http://www.gnu.org/licenses/>.
%
\chapter{Samples}\label{chap:samples}

\lmsbw has a range of samples that show a variety of capabilities and
component configuration options.

Each of these can be found in the \texttt{samples} directory of the
\lmsbw distribution, and each has a script for running them in the
default configuration as well as a \texttt{README} file.


\section{generate}\label{samples:generate}

This sample creates an artificial source tree complete with a
recursive \make build system \emph{and} \lmsbw configuration files
that enables the comparison of the difference in speed between a
recursive \make and the same build wrapped with \lmsbw.

It's just a simple mocked source tree with no complex dependencies,
but the default configuration does produce $90,955$ source files in
$1,856$ directories.

The results of running this sample on the test system can be seen in
figure~\tabref{samples:generate-results}.

\begin{figure}[tb]
  \hrulefill\vspace{10pt}
  \begin{tabularx}{\linewidth}{|r|c|l|l|X|}
    \hline Test & Parallel  & \make & \lmsbw & Description \\
    \hline full build  & 4 & $\approx$ 00:18:00   & $\approx$ 00:18:00 & Full, from scratch, build \\
    \hline \nullbuild  & 4 & 00:00:02.12 & 00:00:00.80 & Build after full build \\
    \hline \nullbuild  & 1 & 00:00:13.96 & 00:00:02.44 & Build after full build \\
    \hline incremental & 4 & 00:00:02.17 & $\approx$ 00:00:01 & Touch file, rebuild \\
    \hline incremental & 1 & 00:00:14.75 & $\approx$ 00:00:04 & Touch file, rebuild \\
    \hline
  \end{tabularx}
  \caption{\texttt{generate} Sample Results}\label{samples:generate-results}
  \hrulefill
\end{figure}


\section{hello-world}

The \texttt{hello-world} sample is a very simple example showing how a
build process can be wrapped with very little work.

Rather than having any component configuration files, this sample
declares the components directly in the configuration loading
function.

\section{Linux Kernel}\label{samples:linux-kernel}

This is a prime example of what \lmsbw can do for an existing,
complex, build system: follow the instructions in the associated
README file to measure how much time \lmsbw will save.  The results of
running this sample on the test system can be seen in
figure~\tabref{samples:linux-kernel-results}.

\begin{figure}[tb]
  \hrulefill\vspace{10pt}
\begin{tabularx}{\linewidth}{|r|c|l|l|X|}
  \hline Test & Parallel  & \make & \lmsbw & Description \\
  \hline full build  & 4 & 00:25:00 & 00:25:26 & Full, from scratch, build \\
  \hline \nullbuild  & 4 & 00:00:61 & 00:00:00 & Build after full build \\
  \hline \nullbuild  & 1 & 00:02:01 & 00:00:01 & Build after full build \\
  \hline incremental & 4 & 00:00:61 & 00:00:62 & Touch file, rebuild \\
  \hline incremental & 1 & 00:02:00 & 00:02:07 & Touch file, rebuild \\
  \hline
\end{tabularx}
\caption{Linux Sample Results}\label{samples:linux-kernel-results}
\hrulefill
\end{figure}

Figure~\tabref{samples:linux-kernel-results} clearly shows that \lmsbw
won't help productivity when working exclusively on the Linux kernel,
but if the Linux kernel is only one component of a large project, it
should be clear how much time is saved by not recursing into the Linux
build process when no work needs to be performed.

\section{lite-consumer-pro}\label{samples:lite-consumer-pro}

This sample aims to show how to configure a wrapped build that has
several different SKUs.  In this case, a \emph{lite}, \emph{consumer}
and \emph{pro} SKU.  The component configuration files associated with
the individual SKUs are configured using symbolic links to the actual
configuration files.

\section{override}\label{samples:override}

This sample quickly shows how to override both the \emph{build} and
\emph{install} rules for a component that is being wrapped.

This is handy if the wrapped component does not have a compliant build
process~(\xref{wrap:component-makefile}).  For example, if the wrapped
\makefile did not have an \emph{install} rule, it could be overridden
to manually install the needed files into \destdir.

\section{source-api}

This sample provides an excellent example of how to declare a \destdir
directory as containing an API; if any file in that directory is
changed, then directly dependent components will be automatically
rebuilt\footnote{\lmsbw will enter the component's build process.  It
  is, obviously, the responsibility of that build process to have the
  proper dependencies written so that API changes are detected and a
  minimal incremental build is performed.}, even if otherwise
unnecessary.  \xref{intro:build-component-criteria}~describes the
criteria used to determine if a component needs to be rebuilt.

\section{sports}\label{samples:sports}

This sample showcases a component configuration mechanism that does
not use symbolic links.  Rather, it loads all the configuration files
and includes or excludes them through the use of \make variables.

This mechanism is really only needed on systems that do not support
symbolic links.

\section{startup-time}\label{samples:startup-time}

This sample allows the time to load any number of mocked component
configuration files to be measured. It's purpose is to provide a rough
estimate of the load-time overhead of using \lmsbw on a project.

The time to load all configuration files also includes the time to
generate all the build-time \make rules, and thus the time produced by
this sample \emph{is} the entire startup overhead -- \make will
immediately begin building the project after the configuration files
are loaded.

The results of running this sample on the test system can be seen in
figure~\tabref{samples:startup-time-results}.  The third and fourth
columns are taken from the \texttt{top} utility, and reflect the
amount of virtual memory and resident memory consumed by the process,
respectively.

\begin{figure}[tb]
  \hrulefill\vspace{10pt}
  \begin{center}
    \begin{tabular}{|c|c|c|c|}
      \hline configuration files  & startup time in seconds & virt Mb & res Mb \\
      \hline   1 & $\approx$ 0.14  & 16.132 & 1.448 \\
      \hline  10 & $\approx$ 0.33  & 16.256 & 1.588 \\
      \hline  25 & $\approx$ 0.59  & 16.504 & 1.824 \\
      \hline  50 & $\approx$ 1  & 16.980 & 2.292 \\
      \hline 100 & $\approx$ 1  & 17.824 & 3.044 \\
      \hline 200 & $\approx$ 3  & 19.658 & 4.708 \\
      \hline 300 & $\approx$ 5  & 21.236 & 6.396 \\
      \hline 400 & $\approx$ 8  & 22.952 & 8.292 \\
      \hline 500 & $\approx$ 10 & 24.700 & 9.976 \\
      \hline
    \end{tabular}
  \end{center}
\caption{\texttt{startup-time} Results}\label{samples:startup-time-results}
  \hrulefill
\end{figure}

It should noted that the task of loading configuration files is
serialized, and is not affected by any amount of parallelization.

\section{build-output-download}

The \texttt{build-output-download} sample provide an extensive example
of how to locally implement a system facilitating the download of
build output.  Downloading build output can help speed up local
developer builds.  Examine the README file for details about using the
sample.
