% Copyright (c) 2012 Taylor Hutt, Logic Magicians Software
%
% This program is free software: you can redistribute it and/or
% modify it under the terms of the GNU General Public License as
% published by the Free Software Foundation, either version 3 of the
% License, or (at your option) any later version.
%
% This program is distributed in the hope that it will be useful, but
% WITHOUT ANY WARRANTY; without even the implied warranty of
% MERCHANTABILITY or FITNESS FOR A PARTICULAR PURPOSE.  See the GNU
% General Public License for more details.
%
% You should have received a copy of the GNU General Public License
% along with this program.  If not, see <http://www.gnu.org/licenses/>.
%
\chapter{Samples}\label{chap:samples}

\lmsbw has a range of samples that show a range of capabilities and
component configuration options.

Each of these can be found in the \texttt{samples} directory of the
\lmsbw distribution, and each has a script for running them in the
default configuration as well as a \texttt{README} file.


\section{generate}\label{samples:generate}

This sample creates an artifical source tree complete with a recursive
\make build system \emph{and} \lmsbw configuration files which allows
you to actually compare the speed difference between a recursive \make
and the same build wrapped with \lmsbw.

Although it's just a simple, dummied-up, source tree with no complex
dependencies, but the default configuration does produce 90,955 source
files in 1856 directories.

Here is a table of results of a run of this sample on the system used
for testing \lmsbw:

\begin{tabularx}{\linewidth}{|r|c|l|l|X|}
  \hline Test & Parallel  & \make & \lmsbw & Description \\
  \hline full build  & 4 & $\approx$ 00:18:00   & $\approx$ 00:18:00 & Full, from scratch, build \\
  \hline \nullbuild  & 4 & 00:00:02.12 & 00:00:00.80 & Build after full build \\
  \hline \nullbuild  & 1 & 00:00:13.96 & 00:00:02.44 & Build after full build \\
  \hline incremental & 4 & 00:00:02.17 & $\approx$ 00:00:01 & Touch file, rebuild \\
  \hline incremental & 1 & 00:00:14.75 & $\approx$ 00:00:04 & Touch file, rebuild \\
  \hline
\end{tabularx}

\section{hello-world}

The \texttt{hello-world} sample is a very simple example of how you
can wrap a build process with very little work.

Rather than having any component configuration files, this sample
declares the components directly in the configuration loading
function.

\section{Linux Kernel}

The Linux Kernel sample is a prime example of what \lmsbw can do
for an existing, real, build system.

Follow the instructions in the associated README file and measure for
yourself how much time \lmsbw will save.

\begin{tabularx}{\linewidth}{|r|c|l|l|X|}
  \hline Test & Parallel  & \make & \lmsbw & Description \\
  \hline full build  & 4 & 00:25:00 & 00:25:26 & Full, from scratch, build \\
  \hline \nullbuild  & 4 & 00:00:61 & 00:00:00 & Build after full build \\
  \hline \nullbuild  & 1 & 00:02:01 & 00:00:01 & Build after full build \\
  \hline incremental & 4 & 00:00:61 & 00:00:62 & Touch file, rebuild \\
  \hline incremental & 1 & 00:02:00 & 00:02:07 & Touch file, rebuild \\
  \hline
\end{tabularx}

If you are working exclusively on the Linux kernel, \lmsbw won't
really help your productivity, but if the Linux kernel is only one
part of a larger project, you can very easily see how much time is
saved by not recursing into the Linux build process when no work needs
to be performed.

\section{lite-consumer-pro}

This sample aims to show you how to configure a wrapped build that has
several different SKUs.  In this case, a \emph{lite}, \emph{consumer}
and \emph{pro} SKU.  The component configuration files associated with
the individual SKUs are configured using symlinks to the actual files.

\section{override}

This sample quickly shows how you can override both the \emph{build}
and \emph{install} rules for a component that is being wrapped.

This is handy if your wrapped build process does not have a compliant
(\xref{wrap:component-makefile}) build process.  For example, if the
wrapped \makefile did not have an \emph{install} rule, you could
override it and manually install the needed files into \destdir.

\section{source-api}

This sample provides an excellent example of how you can declare a
directory as containing an API; if any file in that directory is
changed, then directly dependent components will be automatically
rebuilt, even if otherwise unnecessary.

\section{sports}

This sample showcases a component configuration mechanism which does
not use symlinks.  Rather, it loads all the configuration files and
includes or excludes them through the use of \make variables.

This mechanism is really only needed in systems which do not suppport
symlinks.

\section{startup-time}\label{samples:startup-time}

This sample allows you to measure the time (and indirectly, \lmsbw \&
\make memory consumption) to load any number of dummied-up component
configuration files.

It's purpose is to give you a rough estimate of the load-time overhead
of using \lmsbw on your project.

The time to load all configuration files also includes the time to
generate all the build-time \make rules: the produced number \emph{is}
the startup overhead -- \make will immediately begin building your
project after the configuration files are loaded.

Below is a table which shows the measurements on the test machine for
various numbers of component configuration files.  The third and
fourth columns are taken from the \texttt{top} utility, and reflect
the amount of virtual memory and resident memory consumed by the
process, respectively.

\begin{tabular}{|c|c|c|c|}
  \hline configuration files  & startup time in seconds & virt Mb & res Mb \\
  \hline   1 & $\approx$ 0.14  & 16.132 & 1.448 \\
  \hline  10 & $\approx$ 0.33  & 16.256 & 1.588 \\
  \hline  25 & $\approx$ 0.59  & 16.504 & 1.824 \\
  \hline  50 & $\approx$ 1  & 16.980 & 2.292 \\
  \hline 100 & $\approx$ 1  & 17.824 & 3.044 \\
  \hline 200 & $\approx$ 3  & 19.658 & 4.708 \\
  \hline 300 & $\approx$ 5  & 21.236 & 6.396 \\
  \hline 400 & $\approx$ 8  & 22.952 & 8.292 \\
  \hline 500 & $\approx$ 10 & 24.700 & 9.976 \\
  \hline
\end{tabular}

One aspect of this table which should hit home is that \lmsbw needs
only ten (10) seconds of overhead to load, configure and begin
processing build rules for a project with 500 individual component
configuration files.  Taking such a small amount of memory for 500
components also leaves lots of address space for actual builds to
occur without having to swap out the main build process.

You should note that the task of loading configuration files is
serialized, and is not affected by any amount of parallelization.

