% Copyright (c) 2012 Taylor Hutt, Logic Magicians Software
%
% This program is free software: you can redistribute it and/or
% modify it under the terms of the GNU General Public License as
% published by the Free Software Foundation, either version 3 of the
% License, or (at your option) any later version.
%
% This program is distributed in the hope that it will be useful, but
% WITHOUT ANY WARRANTY; without even the implied warranty of
% MERCHANTABILITY or FITNESS FOR A PARTICULAR PURPOSE.  See the GNU
% General Public License for more details.
%
% You should have received a copy of the GNU General Public License
% along with this program.  If not, see <http://www.gnu.org/licenses/>.
%
\chapter{Uses of \mtree}\label{chap:mtree}

The \mtree utility is used to check if the contents of a directory
tree have changed; on the first run of mtree, it creates a manifest of
each file in the directory and subsequent runs compare the current
status of the files in the directory with the saved manifest.  If
there are no changes, \mtree exits with a zero return code, otherwise
it exits with a non-zero return code.

\lmsbw uses \mtree to determine if a component is up-to-date with
respect to its source directory, but there are many other ways that
the tool can be used by satellite utilities in any build system.

\begin{itemize}
\item Product Manifest

  Use \mtree to ensure that the final product image has the proper set
  of files, file permissions, file owners and file groups.

\item Component Manifest

  Anything installed in the \destdir of a component is a deliverable.
  If the manifest of a component's \destdir changes, that means that
  its set of deliverables might have also changed.

  By tracking component \destdir manifests, changes to a \destdir will
  trigger action: if a new file is suddenly present, or if an existing
  file is no longer present.

\item Ownership \& Group

  Check that all files in the product have the proper ownership and
  group.

\item Permissions

  Check that all files in the product have the proper permissions.

\end{itemize}
