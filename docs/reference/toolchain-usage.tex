% Copyright (c) 2012 Taylor Hutt, Logic Magicians Software
%
% This program is free software: you can redistribute it and/or
% modify it under the terms of the GNU General Public License as
% published by the Free Software Foundation, either version 3 of the
% License, or (at your option) any later version.
%
% This program is distributed in the hope that it will be useful, but
% WITHOUT ANY WARRANTY; without even the implied warranty of
% MERCHANTABILITY or FITNESS FOR A PARTICULAR PURPOSE.  See the GNU
% General Public License for more details.
%
% You should have received a copy of the GNU General Public License
% along with this program.  If not, see <http://www.gnu.org/licenses/>.
%
\chapter{Toolchain Usage}\label{chap:toolchain-usage}

\lmsbw trivially facilitates the use of custom-built toolchains for
building your project, and if you are producing a commercial product,
then you really should be using such a toolchain to build your
project.

The foremost reason for this is that it becomes trivial to reproduce
any build for your project, at any time and on any machine; creating a
bit-for-bit identical build at some point in the future using the
toolchain on your build machines -- which may have been upgraded -- is
unlikely to be be successful.  Secondarily, everyone on your team will
be building the exact same images, regardless the default toolchain
installed on their system.

\lmsbw has been designed to work well with toolchains that have been
produced by the crosstool-ng program
(\xref{chap:toolchain-configuration}).  This has several implications:

\begin{itemize}
\item Reproducible Builds

  In many industries -- industrial control, for example -- it's
  important to be able rebuild a project in exactly the same manner
  as it was built before.  If you simply use your host's toolchain,
  improvements to the compiler may no longer generate the same code,
  and your project may no longer work correctly.

  However, if you use \lmsbw, because the toolchain has been
  specifically configured and built for your project, your project can
  always be built with that toolchain.

\item Architecture Cross Compilation

  Allowing the ability to switch toolchains enables the ability to
  transparently cross-compile your program to a completely different
  architecture.

  When you couple cross-compilation with the component configuration
  model of \lmsbw, you can easily have a project with a core engine
  which is present in every SKU, and an architecture-specific portion
  which is automatically included when necessary.

\item OS Version Cross Compilation

  To compile for different versions of Linux, you will sometimes need
  to use a different version of \texttt{glibc}.  Since the C library
  is part of the toolchain, this becomes very easy.

\item Different Toolchains for Components

  Sometimes a compiling a component with a different compiler is
  required; \lmsbw makes this easy.
\end{itemize}

\section{Using a Toolchain}

The following steps must be taken to use a toolchain:

\begin{enumerate}
\item Configure \& Build toolchains

  Follow the instruction in \xref{chap:toolchain-configuration} to
  build your toolchains.

\item Specify toolchain root directory

  When executing \lmsbw, you must include the
  \texttt{--toolchain-root} argument.  See
  \xref{usinglmsbw:toolchain-root}.

\item Specify desired toolchain

  There are three hierarchical levels at which the toolchain can be set.

  \begin{itemize}
  \item component

    You can indicate that a component must be built with a specific
    toolchain using the \texttt{component\_attribute\_toolchain}
    (\xref{api:toolchain}) API.

    Once a toolchain has been specified at the component level, the
    component will always be built using that toolchain.

    If a component does not have a toolchain specified, then the
    global toolchain setting is used.

    If the specified toolchain does not exist, \lmsbw will produce an
    error.

  \item global

    To use a specific toolchain to build all components with their
    \emph{reason} set to \texttt{image} (\xref{variables:reason}),
    when executing \lmsbw, include the \texttt{--toolchain} argument.

    See \xref{usinglmsbw:toolchain} for more details on this global
    argument.

    If a global toolchain is not specified, then the toolchain
    installed on the host computer is used.

    If the specified toolchain does not exist, \lmsbw will produce an
    error.

  \item host

    If no toolchain has been specified, then \lmsbw will default to
    using the toolchain installed on the host computer.

    If no build tools are installed on the host, then components will
    fail to build.

  \end{itemize}

\end{enumerate}

\section{Toolchain Variables}

\lmsbw will set the \makefile variables to reference the configured
toolchain's versions of the following programs:

\begin{description}
\item{ADDR2LINE}
\item{AR}
\item{AS}
\item{CC}
\item{CPP}
\item{CXX}
\item{GCOV}
\item{GXX}
\item{LD}
\item{NM}
\item{OBJCOPY}
\item{OBJDUMP}
\item{POPULATE}
\item{RANLIB}
\item{READELF}
\item{SIZE}
\item{SIZE}
\item{STRIP}
\end{description}
