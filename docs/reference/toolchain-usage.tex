% Copyright (c) 2012 Taylor Hutt, Logic Magicians Software
%
% This program is free software: you can redistribute it and/or
% modify it under the terms of the GNU General Public License as
% published by the Free Software Foundation, either version 3 of the
% License, or (at your option) any later version.
%
% This program is distributed in the hope that it will be useful, but
% WITHOUT ANY WARRANTY; without even the implied warranty of
% MERCHANTABILITY or FITNESS FOR A PARTICULAR PURPOSE.  See the GNU
% General Public License for more details.
%
% You should have received a copy of the GNU General Public License
% along with this program.  If not, see <http://www.gnu.org/licenses/>.
%
\chapter{Toolchain Usage}\label{chap:toolchain-usage}

\todo{Using sample toolchains}


\lmsbw trivially facilitates using custom-built toolchains for
building your project: just specify the toolchain root directory
(\xref{usinglmsbw:toolchain-root}), and the name of a subdirectory
inside the root directory (\xref{usinglmsbw:toolchain}).

There are many reasons to use a custom-built toolchain over the one
prosent on your host computer.  The foremost reason for this is that
it becomes posslbe to reproduce any build of your project, at any time
and on any machine; creating a bit-for-bit identical build at some
point in the future using the toolchain on your host -- which may have
been upgraded -- is unlikely to be be successful.  Secondarily,
everyone on your team will be building the exact same images,
regardless the default toolchain installed on their system.

\lmsbw has been designed to work with toolchains produced by the
crosstool-ng program (\xref{chap:toolchain-configuration}).  This has
several implications:

\begin{itemize}
\item Reproducible Builds

  In many industries -- industrial control, for example -- it's
  important to be able rebuild a project in exactly the same manner
  as it was built before.  If you simply use your host's toolchain,
  improvements to the compiler may no longer generate the same code,
  and your project may no longer work correctly.

  However, if you use \lmsbw, because the toolchain has been
  specifically configured and built for your project, your project can
  always be built with that toolchain.

\item Architecture Cross Compilation

  Allowing the ability to switch toolchains enables the ability to
  transparently cross-compile your program to a completely different
  architecture.

  When you couple cross-compilation with the component configuration
  model of \lmsbw, you can easily have a project with a core engine
  which is present in every SKU, and an architecture-specific portion
  which is automatically included when necessary.

\item OS Version Cross Compilation

  To compile for different versions of Linux, you will sometimes need
  to use a different version of \texttt{glibc}.  Since the C library
  is part of the toolchain, this becomes very easy.

\item Different Toolchains for Components

  Sometimes a compiling a component with a different compiler is
  required.
\end{itemize}

\section{Sample Configurations}

\lmsbw provides several \ctng configurations to facilitate quickly
being able to cross compile your projects.  In order to be able to use
these sample configurations, you must have first configured \&
installed \ctng.  Please follow the instructions
in \xref{toolchain-config:prerequisites}
and \xref{toolchain-config:install-ctng}.

Once \ctng has been installed, it's very simple to install these
sample toolchains.

\begin{itemize}
  \item Change directory

    Select a toolchain that you want to build, and chagne to that
    directory.  For example, to build one of the MIPS toolchains:

\begin{footnotesize}
\begin{verbatim}
cd ${LMSBW}/toolchains/mips/be_linux.3.3.4_gcc.4.6.3_eglibc.2.15
\end{verbatim}
\end{footnotesize}

  \item Build

    To build and install the toolchain once you have entered the
    directory:

    \texttt{/opt/crosstool-ng/bin/ct-ng build}

    The toolchains will be placed in \texttt{/opt/toolchains}.
\end{itemize}


\section{Using a Custom Toolchain}

The following steps must be taken to use a toolchain:

\begin{enumerate}
\item Configure \& Build toolchains

  Follow the instruction in \xref{chap:toolchain-configuration} to
  build your toolchains.

\item Specify toolchain root directory

  When executing \lmsbw, you must include the
  \texttt{--toolchain-root} argument.  See
  \xref{usinglmsbw:toolchain-root}.

\item Specify desired toolchain

  There are three levels at which the toolchain can be set for your
  project.

  \begin{itemize}
  \item component

    You can indicate that a component must be built with a specific
    toolchain using the \texttt{component\_attribute\_toolchain}
    (\xref{api:toolchain}) API.

    Once a toolchain has been specified at the component level, the
    component will always be built using that toolchain.

    If the specified toolchain does not exist in the toolchain root
    directory, \lmsbw will produce an error.

    If a component does not have a toolchain specified, the global
    toolchain setting is used.

  \item global

    To use a specific toolchain to build all components with their
    \emph{reason} set to \texttt{image} (\xref{variables:reason}),
    when executing \lmsbw, include the \texttt{--toolchain} argument.

    See \xref{usinglmsbw:toolchain} for more details on this global
    argument.

    If the specified toolchain does not exist in the toolchain root
    directory, \lmsbw will produce an error.

    If a global toolchain is not specified, then the toolchain
    installed on the host computer is used.

  \item host

    If no toolchain has been specified, then \lmsbw will use the
    toolchain installed on the host computer.

    If no build tools are installed on the host, then components will
    fail to build; it is your responsibility to ensure that the
    necessary build tools are installed on your computer.

  \end{itemize}

\end{enumerate}

\section{Toolchain Variables}

\lmsbw will set the \makefile variables to reference the configured
toolchain's versions of the following programs:

\begin{description}
\item{ADDR2LINE}
\item{AR}
\item{AS}
\item{CC}
\item{CPP}
\item{CXX}
\item{GCOV}
\item{GXX}
\item{LD}
\item{NM}
\item{OBJCOPY}
\item{OBJDUMP}
\item{POPULATE}
\item{RANLIB}
\item{READELF}
\item{SIZE}
\item{STRIP}
\end{description}
