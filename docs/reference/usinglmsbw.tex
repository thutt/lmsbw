% Copyright (c) 2012 Taylor Hutt, Logic Magicians Software
%
% This program is free software: you can redistribute it and/or
% modify it under the terms of the GNU General Public License as
% published by the Free Software Foundation, either version 3 of the
% License, or (at your option) any later version.
%
% This program is distributed in the hope that it will be useful, but
% WITHOUT ANY WARRANTY; without even the implied warranty of
% MERCHANTABILITY or FITNESS FOR A PARTICULAR PURPOSE.  See the GNU
% General Public License for more details.
%
% You should have received a copy of the GNU General Public License
% along with this program.  If not, see <http://www.gnu.org/licenses/>.
%
\chapter{Using \lmsbw}

Ease-of-use and flexibility were two important design decisions taken
when creating \lmsbw.  After creating the configuration files to wrap
the project's build process, \lmsbwcmd is used to build it.

This chapter only describes how to execute \lmsbwcmd;
\xref{chap:wrapping} describes how to produce the configuration for
wrapping the build.

\section{Accessing \lmsbwcmd}

The \lmsbwcmd script resides in the \texttt{scripts} directory of the
\lmsbw source tree.

Although it is not required, the most convenient way to use \lmsbwcmd
is to add the \texttt{scripts} directory to the \texttt{PATH}
environment variable.

If modifying \texttt{PATH} is undesirable, just use the full pathname
of \lmsbwcmd when executing.

\section{\lmsbwcmd script}

The \lmsbwcmd script is a wrapper for \make.  It ensures that the
preconditions for using the \lmsbw system are met, and once they are
guaranteed \make is executed with the proper environment, command line
arguments, and targets.

The \lmsbwcmd script takes a number of parameters that are described
in the following sections.

\subsection{\texttt{--help}}

This option causes the script to display a help message and then exit.

\subsection{\texttt{--build-root}}

This required option specifies the root directory where build output
for the configured project (\xref{lmsbw:configuration}) will be
placed.

This directory must not be shared with other projects wrapped by
\lmsbw.

\begin{verbatim}
--build-root /tmp/lmsbw-build-output
\end{verbatim}

\subsection{\texttt{--configuration}}\label{lmsbw:configuration}

This required option allows the specification of the master
configuration file that is used to configure \lmsbw for the wrapped
build process.

\begin{verbatim}
--configuration ~/src/project/project.cfg
\end{verbatim}


\subsection{\texttt{--progress}}\label{usinglmsbw:progress}

In normal operation, \lmsbw is very silent about the activities it is
undertaking.  To be presented with more information about the progress
of the build, add this to the command line.

\begin{verbatim}
--progress
\end{verbatim}

Compare \xref{usinglmsbw:verbose}.

\subsection{\texttt{--parallel}}

This option allows the direct control the number of parallel builds
that \make is allowed to invoke.  When not specified on the command
line, the default is one (1).

This value for this argument is directly used with the \gnumake
\texttt{-j} option.

\begin{verbatim}
--parallel 3
\end{verbatim}

\subsection{\texttt{--verbose}}\label{usinglmsbw:verbose}

This option causes \lmsbw to be much more verbose while running.

\begin{verbatim}
--verbose
\end{verbatim}

This option causes every command executed by the main \makefile to be
printed to the console.

Compare \xref{usinglmsbw:progress}

\subsection{\texttt{--time}}

When this option is specified on the command line, \lmsbw will print
the elapsed time to build each component as each is completed.

\begin{verbatim}
--time
\end{verbatim}

\subsection{\texttt{--toolchain}}\label{usinglmsbw:toolchain}

This option globally specifies the toolchain that is to be used to
compile the project.  All components will be compiled using this
toolchain, unless they specify a toolchain in their configuration
file.

If a toolchain is specified with this argument, then the toolchain
root directory with \texttt{--toolchain-root} must also be specified.

The value provided as an argument to this option must be the name of a
subdirectory of the directory named with \texttt{--toolchain-root}.

See \xref{chap:toolchain-usage} for details on using different
toolchains and \xref{chap:toolchain-configuration} for information
about creating toolchains.

\subsection{\texttt{--toolchain-root}}\label{usinglmsbw:toolchain-root}

This option sets the root directory that contains the toolchains which
can be used to build the project.

See \xref{chap:toolchain-usage} for details on using different
toolchains and \xref{chap:toolchain-configuration} for information
about creating toolchains.

\section{Using \lmsbw targets}\label{lmsbw:target:verbs}

Although \lmsbw is extremely fast, it is sometimes desirable to have
it operate on only one component, or to provide further information
about the wrapped build.  The following sections describe methos of
interacting with a wrapped build using \lmsbw.

\subsection{\texttt{all}}\label{lmsbw:target:all}

This is the default target, and it will cause the wrapped build to be
executed in its entirety.

\subsection{\texttt{mtree}}

This target will build, install and show the location of the resulting
\mtree utility executable.

The output will appear in this easily parsed format:

\begin{verbatim}
mtree:/tmp/hello/lmsbw/utilities/bin/mtree
\end{verbatim}

\subsection{\texttt{targets}}

The \texttt{targets} target causes \lmsbw to produce a report of all
top-level targets that are accessible to the user.

The format of the \texttt{target} report is shown in figure
\tabref{usinglmsbw:targets-verb}.

\begin{figure}[tbh]
\hrulefill
\begin{footnotesize}
\begin{verbatim}
                      all : Perform a full build
                    mtree : Install and show mtree executable path
                  targets : Produce this report
               components : List all components in project
      prerequisite-report : Show declared component prerequisites
         dependent-report : Show computed component dependencies
     measure-startup-time : Measure time to load configruation files
 measure-memory-footprint : Measure lmsbw memory overhead
     report[.<component>] : Generate component reports
    install[.<component>] : Build & install components
      clean[.<component>] : Clean build output for components
        log[.<component>] : Display component build logs
    destdir[.<component>] : Shows the intermediate install directory
   builddir[.<component>] : Shows the build directory
 installdir[.<component>] : Shows the install directory
\end{verbatim}
\end{footnotesize}
\hrulefill
\caption{\texttt{targets} verb}\label{usinglmsbw:targets-verb}
\end{figure}

The \texttt{[.<component>]} notation means that this target can also
be executed on a specific component of the wrapped build.  To
determine which components exist in the build, see
\xref{lmsbw:target:components}.

\subsection{\texttt{components}}\label{lmsbw:target:components}

The \texttt{components} target causes \lmsbw to produce a report of all
the components present in the project build.  The report format is
shown in figure \tabref{usinglmsbw:components-verb}.

\begin{figure}[tbh]
\hrulefill
\begin{verbatim}
            goodbye-world : Goodbye World example
              hello-world : Hello World example
\end{verbatim}
\hrulefill
\caption{\texttt{components} verb}\label{usinglmsbw:components-verb}
\end{figure}

\subsection{\texttt{prerequisite-report}}

The \texttt{prerequisite-report} target will produce a report showing
each component and its declared prerequisites.

This output format, shown in figure
\tabref{usinglmsbw:prerequisites-verb}, could be used to produce a
graph of the project dependencies.

\begin{figure}[tbh]
\hrulefill
\begin{verbatim}
goodbye-world: hello-world
hello-world:
\end{verbatim}
\hrulefill
\caption{\texttt{prerequisites-report} verb}\label{usinglmsbw:prerequisites-verb}
\end{figure}


\subsection{\texttt{dependent-report}}

The \texttt{dependent-report} target will produce a report showing
each component and the components which directly depend on it.

This output format, shown in \tabref{usinglmsbw:dependents-verb},
could be used to produce a graph of the project dependencies.

Another use would be to determine which components will be directly
affected by API changes.

\begin{figure}[tbh]
\hrulefill
\begin{verbatim}
goodbye-world:
hello-world: goodbye-world
\end{verbatim}
\hrulefill
\caption{\texttt{dependent-report} verb}\label{usinglmsbw:dependents-verb}
\end{figure}


\subsection{\texttt{measure-startup-time}}\label{usinglmsbw:measure-startup-time}

This target is used to measure the startup overhead of \lmsbw.

This target will load all the configuration files, generate all the
rules for building the project, and then immediately exit.

\subsection{\texttt{measure-memory-footprint}}

This target is used to measure the memory overhead of \lmsbw.

This target is the same as \xref{usinglmsbw:measure-startup-time}, but
instead of exiting, it will wait until the program is killed with
\texttt{ctrl-c}.

It is expected that while the program is waiting, memory consumption
of the \make process started by \lmsbw will be measured.

\subsection{\texttt{report}}\label{usinglmsbw:report}

The \texttt{report} target will produce a report about each wrapped
component in the build.  The report, for each component, will look
as shown in figure \tabref{usinglmsbw:component-report}.

\begin{landscape}
\begin{figure}
\hrulefill
\begin{small}
\begin{verbatim}
hello-world|Component   : hello-world
hello-world|  config    : /lms/build-process/samples/hello-world/hello-world.cfg
hello-world|  desc      : Hello World example
hello-world|  kind      : source
hello-world|  reason    : image
hello-world|  source    : /lms/build-process/samples/hello-world/src
hello-world|  prereq    :
hello-world|  direct dep: goodbye-world
hello-world|  config    : /lms/build-process/samples/hello-world/hello-world.cfg
hello-world|  toolchain :
hello-world|  api       : /usr/include/hello-world
hello-world|Install Root: /tmp/hello/target/install/35f228d9be0694a19de8666fbcdbd80d
hello-world|Build Root  : /tmp/hello/target/hello-world/8a8933dab8deebe52e30886da1ef6ee2
hello-world|  build     : /tmp/hello/target/hello-world/8a8933dab8deebe52e30886da1ef6ee2/build
hello-world|  destdir   : /tmp/hello/target/hello-world/8a8933dab8deebe52e30886da1ef6ee2/destdir
hello-world|Targets     :
hello-world|  install   : install
hello-world|  build     : build
\end{verbatim}
\end{small}
\hrulefill
\caption{Component Report}\label{usinglmsbw:component-report}
\end{figure}
\end{landscape}

\subsection{\texttt{install}}

The \texttt{install} target is used by the default target
(\xref{lmsbw:target:all}) to build and install each component.

\subsection{\texttt{clean}}

The \texttt{clean} target is used to clean out the entire build
directory, or the build directory of a specific component.

\subsection{\texttt{log}}

The \texttt{log} target is used to display all the build logs, or the
build log of a specific component, to the console.

The build log for each component contains all the output produced when
recursively invoking the build process of that component.

If the build fails, the component that failed to build will be shown
on the console; use the \texttt{log} verb to view the actual output of
building that component.

\subsection{\texttt{destdir}}

Because \lmsbw determines the location of the \destdir based on global
configuration options, this target will print out the \destdir for the
current configuration.

Since \destdir contains the install image of the component, knowing
the \destdir location is useful to perform any post-build processing
on the component -- for example, running a debugger on the program or
making sure that all deliverables are installed correctly.

The easily parsed output format is shown in
\tabref{usinglmsbw:destdir-verb}.

\begin{figure}[tbh]
\hrulefill
\begin{scriptsize}
\begin{verbatim}
hello-world:/tmp/hello/target/hello-world/8a8933dab8deebe52e30886da1ef6ee2/destdir
goodbye-world:/tmp/hello/host/goodbye-world/917400cac32350dc041660456b128124/destdir
\end{verbatim}
\end{scriptsize}
\hrulefill
\caption{\texttt{destdir} verb}\label{usinglmsbw:destdir-verb}
\end{figure}

\subsection{\texttt{builddir}}

Because \lmsbw determines the location of the build directory based on
global configuration options, this target will print out the build
directory name for the current configuration.

Knowing the build directory name to post process portions of the build
-- for example, disassembling an object file -- can be helpful.

The easily parsed output format is shown in
\tabref{usinglmsbw:builddir-verb}.

\begin{figure}[tbh]
\hrulefill
\begin{scriptsize}
\begin{verbatim}
hello-world:/tmp/hello/target/hello-world/8a8933dab8deebe52e30886da1ef6ee2/build
goodbye-world:/tmp/hello/host/goodbye-world/917400cac32350dc041660456b128124/build
\end{verbatim}
\end{scriptsize}
\hrulefill
\caption{\texttt{builddir} verb}\label{usinglmsbw:builddir-verb}
\end{figure}

\subsection{\texttt{installdir}}\label{usinglmsbw:installdir}

Because \lmsbw determines the location of the install directory based
on global configuration options, this target will print out the
install directory name for the current configuration.

Knowing the install directory name to post process portions of the
build -- for example, creating \& checking a manifest of files that
are installed can be helpful; if the manifest does not have the same
set of files for each official build, then the \bni team needs to
investigate why the product has changed.

The easily parsed output format is shown in
\tabref{usinglmsbw:installdir-verb}.

\begin{figure}[tbh]
\hrulefill
\begin{footnotesize}
\begin{verbatim}
hello-world:/tmp/hello/target/install/35f228d9be0694a19de8666fbcdbd80d
goodbye-world:/tmp/hello/host/install/35f228d9be0694a19de8666fbcdbd80d
\end{verbatim}
\end{footnotesize}
\hrulefill
\caption{\texttt{installdir} verb}\label{usinglmsbw:installdir-verb}
\end{figure}

