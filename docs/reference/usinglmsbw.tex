% Copyright (c) 2012 Taylor Hutt, Logic Magicians Software
%
% This program is free software: you can redistribute it and/or
% modify it under the terms of the GNU General Public License as
% published by the Free Software Foundation, either version 3 of the
% License, or (at your option) any later version.
%
% This program is distributed in the hope that it will be useful, but
% WITHOUT ANY WARRANTY; without even the implied warranty of
% MERCHANTABILITY or FITNESS FOR A PARTICULAR PURPOSE.  See the GNU
% General Public License for more details.
%
% You should have received a copy of the GNU General Public License
% along with this program.  If not, see <http://www.gnu.org/licenses/>.
%
\chapter{Using \lmsbw}

Ease-of-use and flexibility were two important design decisions taken
when buiding \lmsbw.  Besides creating the configuration files to wrap
your project's build process, you also must be able to execute the
main tool: \lmsbw.  This chapter only describes how to execute \lmsbw;
\xref{chap:wrapping} describes how to produce the configuration for
wrapping your build.

\section{Accessing \lmsbw}

The \lmsbw script resides in the \texttt{scripts} directory of the
\lmsbw source tree.

Although it is not required, the most convenient way to use \lmsbw is
to add the \texttt{scripts} directory to your \texttt{PATH}.

\section{\lmsbw script}

The \lmsbw script is a wrapper for \make.  It ensures that the
preconditions for using the \lmsbw system are met, and once they are
guaranteed \make is executed with the proper set of environment,
command line arguments and targets.

When successfully invoked, \lmsbw will begin building your wrapped
project immediately.

\subsection{\texttt{--help}}

This option causes the script to display a help message and then exit.

\subsection{\texttt{--build-root}}

This option, which is required, allows you to specify the root
directory where build output should be placed.  Build output for the
configured project (\xref{lmsbw:configuration}) will be placed
into the named directory.

This directory should not be shared with other projects wrapped by
\lmsbw.

\begin{verbatim}
--build-root /tmp/lmsbw-build-output
\end{verbatim}

\subsection{\texttt{--configuration}}\label{lmsbw:configuration}

This option, which is required, allows you to specify the file which
is used to configure \lmsbw to wrap your build process.

\begin{verbatim}
--configuration ~/lmsbw/samples/hello-world/hello-world.cfg
\end{verbatim}


\subsection{\texttt{--debug}}

This option, which is not required, enables debugging of the \lmsbw
system.

This option should not be used under normal circumstances.

\begin{verbatim}
--debug
\end{verbatim}

\subsection{\texttt{--progress}}

In normal operation, \lmsbw is very silent about the activities it is
undertaking.  If you want to be presented with more information about
the progress of the build, add this to the command line.

\begin{verbatim}
--progress
\end{verbatim}

\subsection{\texttt{--parallel}}

This option allows you to directly control the number of parallel
builds that \make is allowed to invoke.  When not specified on the
command line, the default is one (1).

This value for this argument is directly used with the \gnumake
\texttt{-j} option.

\begin{verbatim}
--parallel 3
\end{verbatim}


\subsection{\texttt{--tarball-repository}}\texttt{Remove this
  documentation, and remove references in the source.}

This option, which is not required, allows you to specify the
directory where downloaded tarballs should be stored.

This option can be shared between all builds wrapped with \lmsbw; if
the tarball already exists in the repository, it will not be loaded.

\textbf{NOTE: This option is for use with a feature that has not yet
  been implemented.}

\begin{verbatim}
--tarball-repository /tmp/lmsbw-tarballs
\end{verbatim}

\subsection{\texttt{--verbose}}

This option causes \lmsbw to be much more verbose while running.

\begin{verbatim}
--verbose
\end{verbatim}

This option causes every command executed by the main \makefile to be
printed to the console.

\subsection{\texttt{--time}}

When this option is specified on the command line, \lmsbw will print
the elapsed time to build each component as each is completed.

\begin{verbatim}
--time
\end{verbatim}


\section{Using \lmsbw \make targets}

Although \lmsbw is extremely fast, it is sometimes desirable to have
it operate on only one component, or to provide further information
about the wrapped build.

\subsection{\texttt{all}}\label{lmsbw:target:all}

This is the default target, and it will cause your wrapped build to be
executed in its entirety.

\subsection{\texttt{targets}}

The \texttt{targets} target causes \lmsbw to produce a report of all
top-level targets that are accessible to the user.

The \texttt{target} report will have this format:

\begin{verbatim}
                      all : Perform a full build
                  targets : Produce this report
     report[.<component>] : Generate component reports
    install[.<component>] : Build & install components
      clean[.<component>] : Clean build output for components
        log[.<component>] : Display component build logs
      prerequisite-report : Show declared component prerequisites
         dependent-report : Show computed component dependencies
\end{verbatim}

The \texttt{[.<component>]} notation means that this target can also
be executed on a specific component of the wrapped build.  To
determine which components exist in the build, see
\xref{lmsbw:target:components}.

\subsection{\texttt{report}}

The \texttt{report} target will produce a report about each wrapped
component in the build.  The report, for each component, will look
something like this:

\begin{footnotesize}
\begin{verbatim}
hello-world|Component   : hello-world
hello-world|  config    : /lms/build-process/samples/hello-world/hello-world.cfg
hello-world|  desc      : Hello World example
hello-world|  kind      : source
hello-world|  reason    : image
hello-world|  source    : /lms/build-process/samples/hello-world/src
hello-world|  prereq    :
hello-world|  direct dep: goodbye-world
hello-world|  config    : /lms/build-process/samples/hello-world/hello-world.cfg
hello-world|  toolchain :
hello-world|  api       : /usr/include/hello-world
hello-world|Install Root: /tmp/hello/target/install/35f228d9be0694a19de8666fbcdbd80d
hello-world|Build Root  : /tmp/hello/target/hello-world/8a8933dab8deebe52e30886da1ef6ee2
hello-world|  build     : /tmp/hello/target/hello-world/8a8933dab8deebe52e30886da1ef6ee2/build
hello-world|  destdir   : /tmp/hello/target/hello-world/8a8933dab8deebe52e30886da1ef6ee2/destdir
hello-world|Targets     :
hello-world|  install   : install
hello-world|  build     : build
\end{verbatim}
\end{footnotesize}

\subsection{\texttt{install}}

The \texttt{install} target is used by the default target
(\xref{lmsbw:target:all}) to build and install each component.

\subsection{\texttt{clean}}

The \texttt{clean} target is used to clean out the entire build
directory, or the build directory of a specific compoment.

\subsection{\texttt{log}}

The \texttt{log} target is used to display all the build logs, or the
build log of a specific component, to the console.

The build log for each component contains all the output produced when
recursively invoking the build process of that component.

If the build fails, the component that failed to build will be shown
on the console, and then you can use the \texttt{log} target to view
the actual output of building that component.


\subsection{\texttt{prerequisite-report}}

The \texttt{prerequisite-report} target will produce a report showing
each component and its declared prerequisites.

This output could be used to produce a graph of the project dependencies.

\begin{verbatim}
goodbye-world: hello-world
hello-world:
\end{verbatim}

\subsection{\texttt{dependent-report}}

The \texttt{dependent-report} target will produce a report showing
each component and the components which directly depend on it.

This output could be used to produce a graph of the project
dependencies.

Another use would be to determine which components will be directly
affected by API changes.

\begin{verbatim}
goodbye-world:
hello-world: goodbye-world
\end{verbatim}

\subsection{\texttt{components}}\label{lmsbw:target:components}

The \texttt{compnents} target causes \lmsbw to produce a report of all
the components present in your project build.  The report will take
this form:

\begin{verbatim}
            goodbye-world : Goodbye World example
              hello-world : Hello World example
\end{verbatim}
