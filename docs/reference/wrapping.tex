% Copyright (c) 2012 Taylor Hutt, Logic Magicians Software
%
% This program is free software: you can redistribute it and/or
% modify it under the terms of the GNU General Public License as
% published by the Free Software Foundation, either version 3 of the
% License, or (at your option) any later version.
%
% This program is distributed in the hope that it will be useful, but
% WITHOUT ANY WARRANTY; without even the implied warranty of
% MERCHANTABILITY or FITNESS FOR A PARTICULAR PURPOSE.  See the GNU
% General Public License for more details.
%
% You should have received a copy of the GNU General Public License
% along with this program.  If not, see <http://www.gnu.org/licenses/>.
%
\chapter{Wrapping Your Build}\label{chap:wrapping}

Determining the best way to wrap your project's build is important to
the maintainability and speed of the build process as a whole.  This
chapter will guide you with some suggestions for setting up and
creating your component configuration files.

\section{Configuration Techniques}

\lmsbw provides its greatest benefits when wrapping large projects
consisting of many different sub-projects and possibly many different
product configurations.  The uncertainty that aries about component
configuration from situations like these is understandable; \lmsbw is
a new tool for you.

Here are a few tenets to help guide you to the right configuration
choices:

\begin{itemize}
  \item \makefile maintenance is for the \emph{development} teams.

    Since \lmsbw strictly defines an interface -- an API -- for
    \makefile implementation, it easily follows that the developers
    responsible for a sub-project should be responsible for writing a
    conformant \makefile for that sub-project.  After all, they are
    the ones familiar with the interrelationships between the source
    modules.

    If needed, the \bni team can become involved to help with
    performance or conformance issues.

  \item Overall build \emph{process} maintainance is for the
    \emph{\bni} team.

    The \bni team is not responsible for sub-project \makefile
    maintenance, but they are responsible for the overall
    infrastructure of the build process as a whole.

    They will be responsible for the automated build machines,
    delivery of build output, and to a large part, the overall
    structure and maintenance of the output produced by the build
    process.

  \item High-level product configuration is for \emph{product
    management}.

    The product configuration -- which sub-projects are included into
    each SKU produced by your company -- is a task owned by the
    product management team.  Their input for configuration of \lmsbw
    is if your project is big enough to have multiple SKUs.

  \item Build process configuration is an amalgam of roles

    The configuration of the build process requires the cooperation of
    the product management team (to determine what is included and
    what is excluded), the development team (to create \& maintain a
    \makefile for their sub-project) and the \bni team (to create the
    actual configuration that \lmsbw will use).
\end{itemize}

Cooperation is important in all of this; no one likes change,
especially when the final results are nebulous until everything is
finalized and working.

\subsection{Configuration File Directory}

The basic recommendation for \lmsbw component configuration is to
place all your configuration files into the same directory.  This
centralizes all the work and makes global changes easier.  Your
\texttt{load-configuration} function can simply load all the component
configuration files from the common directory.

\subsection{Symlink to Configuration File}

With a more complicated product structure, perhaps one with multiple
SKUs and different componetns in each SKU, you can still keep all your
configuration files in a single directory, but on top of that you will
have \emph{configuration} directories for each distinct product.
These \emph{configuration} directories will be populated with relative
symlinks to the original configuration file; see the
\texttt{lite-consumer-pro} sample for an example of how this is
performed.\todo{What if the OS does not support symlinks?}

\subsection{Conditional Declaration}

If your operating system does not support symbolic links, you can
export an environment variable from your shell, include \emph{all}
your component configuration files, and use the facilities of \gnumake
to conditionally include only the desired components.

See the \texttt{sports} sample for an example of this configuration
strategy.

\section{Converting a Build}

\section{Source Components}
\subsection{\texttt{declare\_source\_component}}
