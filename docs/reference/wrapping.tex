% Copyright (c) 2012 Taylor Hutt, Logic Magicians Software
%
% This program is free software: you can redistribute it and/or
% modify it under the terms of the GNU General Public License as
% published by the Free Software Foundation, either version 3 of the
% License, or (at your option) any later version.
%
% This program is distributed in the hope that it will be useful, but
% WITHOUT ANY WARRANTY; without even the implied warranty of
% MERCHANTABILITY or FITNESS FOR A PARTICULAR PURPOSE.  See the GNU
% General Public License for more details.
%
% You should have received a copy of the GNU General Public License
% along with this program.  If not, see <http://www.gnu.org/licenses/>.
%
\chapter{Wrapping A Build}\label{chap:wrapping}

Determining the best way to wrap a project's build is important to the
maintainability and speed of the build process as a whole.  This
chapter provides guidance for setting up and creating the master
configuration, component configurations and the API requirements of a
component \makefile.

The following is a breakdown of the steps that should be taken to
wrap a build:

\begin{description}
\item{Defining Build Process Roles}

  Maintaining a build process for a large system transcends the \bni
  team; managers and developers must also become involved stake
  owners.  Definition, and acceptance, of roles for all players is an
  important part of success.

\item{Master Configuration Creation}

  The master configuration file sets up \lmsbw to load all component
  configurations that are part of the project.

\item{Component Creation}

  Deciding how to split a project to create components and component
  configuration files will have a direct affect on the quality of the
  wrapped build.

\item{Component \makefile Creation}

  Each \makefile associated with a component must meet the compliance
  requirements of \lmsbw.

\end{description}

\section{Defining Build Process Roles}\label{wrap:build-process-roles}

\lmsbw provides its greatest benefits when wrapping large projects
consisting of many different sub-projects and possibly many different
product configurations.  The uncertainty that arises about component
configuration from situations like these is understandable; \lmsbw is
a newly introduced tool.

Here are a few tenets of guidance on configuration choices:

\begin{description}
  \item High-level Product Configuration

    This job is most well-suited to \emph{product management}.

    The product configuration -- which sub-projects are included into
    each SKU produced by the project -- is a task owned by the product
    management team.

  \item Build Process Configuration

    This aspect is an amalgam of roles: the configuration of the build
    process requires the cooperation of the product management team
    (to determine what is included and what is excluded), the
    development teams (to create \& maintain a \makefile for each
    sub-project) and the \bni team (to create the actual configuration
    that \lmsbw will use).

  \item Build Process Maintenance

    This job is best subsumed by the \emph{\bni} team.  The \bni team
    is not responsible for sub-project \makefile maintenance, but it
    is responsible for the overall infrastructure of the build process
    as a whole.

    It will be responsible for the automated build machines, delivery
    of build output, and to a large part, the overall structure and
    maintenance of the output produced by the build process.

  \item Component \makefile Maintenance

    As the development team responsible for delivering a sub-project
    is most familiar with the interrelationships between source
    modules, it follows that such a team is also responsible for
    producing a corresponding \makefile that is compliant with the
    \lmsbw system for its sub-project.

    If needed, the \bni team can become involved to help with
    performance or conformance issues.

\end{description}

Cooperation is important in all of this; a build process will only be
successful if every participant pulls their weight and strives for the
best outcome.

\section{Master Configuration Creation}\label{wrap:master-configuration}

The master configuration file must do the following:

\begin{itemize}
\item Declare a component loading function.  See
  \xref{variables:load-configuration-function}.
\item Declare the desired \emph{build support}.  See
  \xref{variables:component-build-support}.
\item Provide an implementation for the declared component loading
  function.
\end{itemize}

Loading component configuration files is straightforward, but as
different operating systems have varying file system capabilities, the
decisions on how to set up component configuration files will be
affected.  The following sections provide some suggestions, and some
pointers to sample configurations contained in the \lmsbw
distribution.

\subsection{Directory of Component Configuration Files}\label{wrap:directory}

The basic recommendation for \lmsbw component configuration is to
place all component configuration files into the same directory.  This
centralizes maintenance and makes global configuration changes much
easier.  The \texttt{load-configuration} function can easily load all
the component configuration files from the common directory.

\subsection{Symlinks to Component Configuration Files}

With a more complicated project structure, perhaps one with multiple
SKUs and different components in each SKU, it is still possible to
keep all the configuration files in a single directory, but on top of
that there will be \emph{configuration} directories for each distinct
SKU.

These \emph{configuration} directories will be populated with relative
symbolic links\footnote{Always use relative symbolic links.  Absolute symbolic links
  encode full pathname structure into the build system.} to the
original configuration file; see the \texttt{lite-consumer-pro} sample
(\xref{samples:lite-consumer-pro}) for an example of how this is
performed.

\subsection{Conditional Component Declaration}

For operating systems that do not support symbolic links, one
configuration option is the following:

\begin{itemize}
\item Export an environment variable from the shell which can be used
  to discriminate the product configuration,
  Figure~\xref{wrapping:configuration-using-environment-variables}
  shows how this can be done.

\begin{figure}[b]
\hrulefill
\begin{verbatim}
export PRODUCT=shareware;   # freeware, shareware, commercial
\end{verbatim}
\hrulefill
\caption{Component Configuration using Environment Variables}
\label{wrapping:configuration-using-environment-variables}
\label{usinglmsbw:sourcedir-verb}
\end{figure}

\item Include all the component configuration files, as described in
  \xref{wrap:directory},
\item Use the facilities of \gnumake to conditionally include only the
  desired components using the exported environment variable.
\end{itemize}

See the \texttt{sports} sample (\xref{samples:sports}) for an example
of this configuration strategy.

\subsection{Sample Configuration File}
Figure~\tabref{wrap:sample-master-config} shows the
\texttt{lite-consumer-pro} sample master configuration file.  This
configuration file does the following:

\begin{itemize}
\item{Define Component Loading Function}

  At the beginning of the file, the \texttt{load\_configuration}
  function is declared.  This function is responsible for loading the
  component configuration files for the project.  This sample shows
  how to configure different products by using symbolic links to
  configuration files.

  \begin{itemize}
  \item{Check \texttt{PRODUCT}}

    The \texttt{PRODUCT} environment variable dictates if the
    \texttt{lite}, \texttt{consumer} or \texttt{pro} product is being
    built.  Consequently, the function first checks that
    \texttt{PRODUCT} is set.

    If \texttt{PRODUCT} is not set to a correct value, the build will
    fail with an error message that shows the allowed values for
    \texttt{PRODUCT}.

  \item{Assert configuration directory}

    If  \texttt{PRODUCT} is set, the function next validates that it
    is an allowed value by asserting that a directory exists with the
    \texttt{PRODUCT} name.

  \item{Set \texttt{CURRENT\_CONFIGURATION\_FILE}}

    This variable is used by the component configuration files to
    declare the components (\xref{api:declare-source-component}).

    It contains the full pathname of the configuration file that is
    currently being loaded.

    This value can also be used to derive the source tree directory,
    which is also a required parameter.

  \item{Include component configuration}

    The function loads the component configuration using the
    \texttt{\$(include)} directive, and that causes \lmsbw to generate
    all the rules to maintain the component.

  \end{itemize}

  \item{Declare Component Loading Function to \lmsbw}

    The \texttt{gmsl} \texttt{set} function is used to assign
    \texttt{load\_configuration} to the associative array key,
    \texttt{load-configuration-function}
    (\xref{variables:load-configuration-function}).

  \item{Declare Component Support}

    The project declares that it will be using \texttt{source}
    components by assigning to the key \texttt{component-build-support}
    (\xref{variables:component-build-support}).
\end{itemize}

\begin{landscape}
\begin{figure}
\hrulefill
\begin{verbatim}
define load_configuration
$(call assert,$(PRODUCT), \
    PRODUCT must be one of '$(notdir $(wildcard $(dir $(1))products/*))')
$(call assert_exists,$(dir $(1))products/$(PRODUCT))
$(foreach cf,$(wildcard $(dir $(1))products/$(PRODUCT)/*),	\
          $(eval CURRENT_CONFIGURATION_FILE:=$(cf))		\
          $(eval include $(cf))					\
)
endef


# Set up the LMSBW_configuration associative array.
#
vv:=$(call set,LMSBW_configuration,load-configuration-function,load_configuration)

# Declare the type of components that will be needed.  There is only
# source in this sample, so only load support for 'source' components.
#
vv:=$(call set,LMSBW_configuration,component-build-support,source)
\end{verbatim}
\hrulefill
\caption{Example Master Configuration File}\label{wrap:sample-master-config}
\end{figure}
\end{landscape}

\section{Component Creation}\label{wrap:component-creation}

A component is a set of source files that can be transformed into a
set of deliverables for a project.  Components must be declared to
\lmsbw through a simple API that has been provided; once declared,
\lmsbw takes care of all the administrative details for building.

One good heuristic for determining how to split a system into
components is to start with the top-level directories that are entered
via \make-recursion when using the current build system.  The overall
\nullbuild times will immediately improve, but if each
recursively-invoked \makefile again invokes \make, the incremental
builds can become the bottleneck.

If incremental builds become a bottleneck, the recommended solution is
to \emph{divide \& conquer}: create first-class components from second
or third level directories in the build process.  Each sub-directory
directly assigned to component will have little impact on the
\nullbuild time, but will have a great impact on reducing incremental
build times.  It will have a great impact because fewer directories
will be entered to build changed components.

For each declared source component, \lmsbw will internally generate
\makefile rules that are used to perform the \lmsbw verbs
(\xref{usinglmsbw:target:verbs}) on the component.

The following sections describe how to declare the supported
component types of \lmsbw.

\subsection{\texttt{declare\_source\_component}}

A source component is a set of sources that is directly available in
the project's source tree.

To declare a source component to \lmsbw, the following
information is required:

\begin{itemize}
\item{Component Name}

  A name must be provided that will be used to refer to the component
  when using \lmsbwcmd, or as a prerequisite of other components.

  The name supplied must be a valid \makefile variable name.

  The name must be unique in the project.

\item{Component Description}

  A brief description of the component must must supplied.

\item{Reason for Building Component}

  A component is included in a project for one of two reasons:

  \begin{itemize}
    \item It's part of the delivered product
    \item It's used by the build system to help build the delivered
      product.
  \end{itemize}

  If the component is part of the product, the reason is
  \texttt{image}, and if it's part of the build the reason is
  \texttt{build}.  This choice affects which toolchain is used to
  build the component, and where it is installed in the build tree.
  See \xref{variables:reason}.

\item{Configuration File}

  The full pathname of the configuration file that defines the
  component.  It is included as part of the declaration so that
  changes to the configuration file will force the component build
  process to be entered.

  Since the pathname may not be known until the configuration file
  itself is loaded, it is the responsibility of the
  \texttt{load-configuration-function}
  (\xref{variables:load-configuration-function}) to marshal this
  information to the component configuration file.

  See the \texttt{sports} (\xref{samples:sports}) sample for one
  approach to addressing this issue.

\item{Path to Source Directory}

  The full pathname to the directory containing the sources of the
  component is also necessary.  To be flexible and allow any client of
  the wrapped build to put the software in any location on disk, it is
  incumbent upon the developer to determine the full pathname of the
  source directory at configuration load time.  There are many ways to
  accomplish this, here are a few suggestions:

  \begin{itemize}
    \item Use the \gnumake \texttt{realpath} function and a relative path from
      the configuration file.
    \item Use an environment variable to denote the root of the source
      tree, and append the relative path to the component.

      See the sources associated with \xref{samples:linux-kernel},
      \xref{samples:override} and \xref{samples:sports} for examples
      similar in concept.

      Figure~\tabref{wrap:simple-component-declaration} shows how an
      environment variable can be used to create the source directory
      name.
  \end{itemize}

\item{List of prerequisite components}

  If the component requires one or more other components to have been
  built and installed prior to this component being built, provide
  that information when declaring a component.  \lmsbw will ensure
  that all prerequisite components are built and installed before the
  dependent components are built.
\end{itemize}

Once this information has been gathered, it is possible to declare
components using the \texttt{declare\_source\_component} function
(\xref{api:declare-source-component}).  The following are a few
examples of declaring source components:

\begin{itemize}
\item A component used in the build

Figure~\tabref{wrap:simple-component-declaration} declares a component
named \texttt{compiler}, that is included in the project because it is
used as part of the \texttt{build}.  It's configuration file is named
by a \make variable, and the source is relative to a directory named
by the variable \texttt{PROJECT\_SOURCE\_ROOT}.

\begin{figure}
\hrulefill
\begin{verbatim}
$(call declare_source_component,              \
       compiler,                              \
       Compiler for build system,             \
       build,                                 \
       $(CURRENT_CONFIGURATION_FILE),         \
       $(PROJECT_SOURCE_ROOT)/src/compiler)
\end{verbatim}
\hrulefill
\caption{Simple Source Component Declaration}\label{wrap:simple-component-declaration}
\end{figure}

Any other component that \emph{uses} this compiler in the build
process must list the \texttt{compiler} component as a prerequisite.

\item A component with a prerequisite

Figure~\tabref{wrap:prerequisite-component-declaration} declares a
component, named \texttt{consumer}, with a prerequisite called
\texttt{producer}; \texttt{producer} will always be built before
\texttt{consumer}.  It is included in the project because it is
delivered as part of the project \texttt{image}.  It's configuration
file is named by a \make variable.

\begin{figure}
\hrulefill
\begin{verbatim}
$(call declare_source_component,              \
       consumer,                              \
       Consumes data structure & API,         \
       image,                                 \
       $(CURRENT_CONFIGURATION_FILE),         \
       $(subst sample.cfg,src/consumer,$(1)), \
       producer)
\end{verbatim}
\hrulefill
\caption{Source Component with Prerequisite Declaration}\label{wrap:prerequisite-component-declaration}
\end{figure}
\end{itemize}

An idiom that will be seen in many component configuration files is
the handling of parameter number five (5) -- the source directory
path.  This file containing this code is loaded from the declared
\texttt{load-configuration-function}, and that component
configuration-loading function is passed a single parameter: the full
pathname of the master configuration file itself.

Of course, since the component configuration file is loaded and
evaluated in the context of the \texttt{load-configuration-function},
the full path to the configuration file is still available as
\texttt{\$(1)}\footnote{Provided the configuration file and the source
  directory are in the same directory tree.}.

\begin{figure}
\hrulefill
\begin{small}
\begin{verbatim}
  .../build-process/samples/source-api:
  drwxrwxr-x  3 users users 4096 Jul 26 20:26 .
  drwxr-xr-x 10 users users 4096 Jul 19 19:07 ..
  -rw-rw-r--  1 users users 1664 Jul 21 20:34 sample.cfg
  drwxrwxr-x  4 users users 4096 Jun 27 05:23 src

  .../build-process/samples/source-api/src:
  drwxrwxr-x 4 users users 4096 Jun 27 05:23 .
  drwxrwxr-x 3 users users 4096 Jul 26 20:26 ..
  drwxrwxr-x 2 users users 4096 Jul 14 19:30 consumer
  drwxrwxr-x 2 users users 4096 Jul 21 20:00 producer
\end{verbatim}
\end{small}
\hrulefill
\caption{Directory Structure}\label{wrap:config-structure}
\end{figure}

Figure \tabref{wrap:config-structure} shows that the idiom in argument
five (5) is correct because the project configuration file resides in
the same directory as the component source directory.  The code for
this parameter replaces the configuration file name -- that is known
-- with the name of the source directory for the component.

\subsubsection{Valid Attributes}

The following attributes can be set for source components.

\begin{itemize}
\item{\texttt{component\_attribute\_no\_parallel\_build}}
  (\xref{api:component-attribute-no-parallel-build})
\item{\texttt{component\_attribute\_build\_target}}  (\xref{api:component-attribute-build-target})
\item{\texttt{component\_attribute\_install\_target}} (\xref{api:component-attribute-install-target})
\item{\texttt{component\_attribute\_api}} (\xref{api:component-attribute-api})
\item{\texttt{component\_attribute\_build\_output\_download}} (\xref{api:build-output-download})
\end{itemize}

\section{Component \makefile Creation}\label{wrap:component-makefile}

Each component must have a build process that adheres to the following
rules:

\begin{itemize}
\item \makefile

  Each component declares a directory that contains the source code
  and, implicitly, the build process for it.  The build process must
  be invoked through a \makefile and the \makefile must be present in
  the top-level source directory of the component.

\item Toolchain

  \lmsbw manages the toolchain that is used when building a component,
  and therefore each component's build process should only use the
  standard macros for accessing toolchain tools; for example, use
  \texttt{\$(CC)} rather than \texttt{cc} or \texttt{gcc}.

  The \gnumake reference manual describes all the standard macros that
  are available to access development tools, and
  \xref{toolchain-usage:variables} lists the standard macros provided
  by \lmsbw.

\item Test Execution

  Because \lmsbw facilitates easy cross-compilation to a different
  architecture, no component that is declared to be part of the
  product image (see \xref{variables:reason}) should execute any
  program that is built by the component build\footnote{Unless it is
    known with certainty that the build will never be cross-compiled
    to a different architecture.}.

  Be very careful when testing in a cross-compiled environment; the
  details of this are beyond the scope of this software.

\item \texttt{build} target

  The \makefile must have a target that is invoked to \emph{build} the
  component.  By default, \lmsbw will invoke the component \makefile
  with no target; in other words, the default target of the component
  \makefile is used to build the component.

  The \texttt{component\_attribute\_build\_target} API
  (\xref{api:component-attribute-build-target}) allows the
  specification of the target(s) used to build a individual component.

  \xref{overriding:overriding-build} shows how to override this
  default behavior.

\item \texttt{install} target

  The \makefile must have a target that is invoked to \emph{install}
  the component.  By default, \lmsbw will invoke the component
  \makefile with the target \texttt{install}.

  The \texttt{install} target must \emph{install} public output of the
  build process into the directory denoted by the \makefile variable
  \texttt{DESTDIR}.  \texttt{DESTDIR} is a standard \make variable
  used when installing software.

  To be compliant, a component's build process should be written to
  assume that the directory into which it installs files is the root
  directory.  For example, if the component, \texttt{alpha}, is
  installing header files, it should be written to do something as
  shown in figure~\tabref{wrap:install-target}.

\begin{figure}
\hrulefill
\begin{verbatim}
$(MKDIR) --parents $(DESTDIR)/usr/include/alpha
$(CP) --recursive                          \
     $(LMSBW_C_BUILD_DIRECTORY)/include/*  \
     $(DESTDIR)/usr/include/alpha
\end{verbatim}
\hrulefill
\caption{Installing from \texttt{install} target}\label{wrap:install-target}
\end{figure}

  \lmsbw guarantees that \texttt{DESTDIR} will be defined, and that
  the directory will exist.

  The \texttt{component\_attribute\_install\_target} API
  (\xref{api:component-attribute-install-target}) allows specifying of
  the target used to install an individual component.

  \xref{overriding:overriding-install} shows how to override this
  default behavior.

\end{itemize}

\section{Using a Prerequisite's API}\label{wrap:using-prerequisite-api}

When a component has listed a prerequisite component, and the
prerequisite component has declared an API, access the API directories
by appending the API directory to the value of
\texttt{LMSBW\_C\_INSTALL\_DIRECTORY}.

If, for example, a prerequisite component declared an API at
\texttt{/usr/include/src}, access the include files with:

\begin{footnotesize}
\begin{verbatim}
CFLAGS  := $(LMSBW_C_CFLAGS) -I$(LMSBW_C_INSTALL_DIRECTORY)/usr/include/src
\end{verbatim}
\end{footnotesize}

\section{Using \texttt{build} Components}\label{wrap:using-build-components}

The \texttt{LMSBW\_C\_BUILD\_INSTALL\_DIRECTORY} variable
(\xref{wrap:build-install-directory}) is passed down to the component
build process, and it can be used to access \texttt{build} component
data.

This is the root of the install directory of components that have
their \texttt{reason} (\xref{api:reason}) set to \texttt{build}
(\xref{variables:reason}).

For example, if the project had a \texttt{build} component named
\texttt{check} that installs binaries in \texttt{/usr/bin/check},
access it from the component's \makefile with the following:

\begin{verbatim}
$(LMSBW_C_BUILD_INSTALL_DIRECTORY)/usr/bin/check
\end{verbatim}
