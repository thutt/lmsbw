% Copyright (c) 2012 Taylor Hutt, Logic Magicians Software
%
% This program is free software: you can redistribute it and/or
% modify it under the terms of the GNU General Public License as
% published by the Free Software Foundation, either version 3 of the
% License, or (at your option) any later version.
%
% This program is distributed in the hope that it will be useful, but
% WITHOUT ANY WARRANTY; without even the implied warranty of
% MERCHANTABILITY or FITNESS FOR A PARTICULAR PURPOSE.  See the GNU
% General Public License for more details.
%
% You should have received a copy of the GNU General Public License
% along with this program.  If not, see <http://www.gnu.org/licenses/>.
%
\chapter{Toolchain Configuration}

\section{\texttt{crosstool-ng}}

\subsection{Paths and misc options}
\begin{itemize}
\item{Prefix Directory}

  Set to: \texttt{/opt/toolchains/\$\{CT\_TARGET\}}

\item{Render the toolchain read-only}

  If you will need to make changes to the toolchain, or possibly
  delete it, unselect this option.

\item{Number of parallel jobs}

  This option affects the number of parallel jobs used when building
  the toolchain.  The system will automatically determine a reasonable
  number, but you change change it.

\item{Maximum Allowed load}

  This option affects the total load building the toolchain can put on
  your system.  Adjust for your personal taste.
\end{itemize}

\subsection{Target Options}
\begin{itemize}
\item{Target Architecture}

  Select the desired target architecture.

\item{Endianess}

  If allowed, select the endianess for the selected architecture.

\item{Bitness}

  If allowed, select the bitness for the selected architecture.

\item{ABI}

  If allowed, select the desired ABI for the selected architecture.

\end{itemize}

\subsection{Toolchain Options}
\begin{itemize}
\item{Tuple's vendor string}

  This is a string which can be used to further uniquely identify the
  toolchain.  A tuple is of the form 'arch-vendor-kernel-system' and
  is used to form the name of the tools in the toolchain, as well as
  the toolchain directory name.

  If you intend on creating many different toolchains -- perhaps for
  different compiler versions, or different kernels -- it is
  recommended to createa vendor string which mnemonically
  distinguishes toolchains from one another.

  You could use, for example, the gcc version as the vendor string.
  For a \emph{bare metal} MIPS configuration, you might end up with a
  tuple like this: \texttt{mipsel-gcc\_4.6.3-elf}.

  You may not use dashes in the vendor string.

\end{itemize}

\subsection{Operating System}
\begin{itemize}
\item{Target OS}

Choose the type of OS you wish to target.

If you choose \texttt{linux}, you will have to make several more
choices about the Linux kernel.
\end{itemize}

\subsection{Binary Utilities}

The defaults here are normally sufficient.

\subsection{C compiler}

Choose the compiler version and the desired additional language
support.

\subsection{C-library}

Choose the desired C-library implementation.

\subsection{Debug facilities}

Choose the desired debugging facilities.

\subsection{Companion Libraries}

Choose the desired companion libraries.
